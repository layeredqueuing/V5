%%  -*- mode: latex; mode: outline-minor; fill-column: 108 -*-
%% Title:  lqsim
%%
%% $HeadURL$
%% Original Author:     Greg Franks <greg@sce.carleton.ca>
%% Created:             Tue Jul 18 2006
%%
%%
%% ----------------------------------------------------------------------
%% $Id$
%% ----------------------------------------------------------------------

\chapter{Invoking the Simulator ``lqsim''}
\label{sec:invoking-lqsim}

Lqsim is used to simulate layered queueing networks using the
PARASOL\index{Parasol}~\cite{perf:neilson-91b} simulation system.
Lqsim reads its input from files specified at the command line if
present, or from the standard input otherwise.  By default, output for
an input file \texttt{filename} specified on the command line will be
placed in the file \texttt{filename.out}.  If the \flag{p}{} switch is
used, parseable output\index{output!parseable} will also be written
into \texttt{filename.p}. If XML input\index{input!XML} is used,
results will be written back to the original input file.  This
behaviour can be changed using the \flag{o}{output} switch, described
below.  If several files are named, then each is treated as a separate
model and output will be placed in separate output files.  If input is
from the standard input, output will be directed to the standard
output.  The file name `\texttt{-}' is used to specify standard input.

The \flag{o}{output} option can be used to direct output to the file or directory named \emph{output}
regardless of the source of input.  Output will be XML\index{XML}\index{output!XML} if XML
input\index{input!XML} is used, parseable output if the \flag{p}{} switch is used, and normal output
otherwise; multiple input files cannot be specified.  If \emph{output} is a directory, results will be
written in the directory named \texttt{output}.  Output can be directed to standard output by using
\texttt{-o-} (i.e., the output file name is `\texttt{-}'.)

\section{Command Line Options}
\label{sec:lqsim-options}\index{command line}

\begin{description}
\item[\flag{A}{}, \longopt{automatic}=\emph{run-time[,precision[,skip]]}]~\\
  Use automatic
  blocking\index{automatic blocking}\index{block!automatic} with a
  simulation block
  size\index{block!size}\index{simulation!block}\index{block!simulation}
  of \emph{run-time}\index{run time!simulation}.  The
  \emph{precision}\index{precision!simulation} argument specifies the
  desired mean 95\% confidence level\index{confidence level}.  By
  default, precision is 1.0\%.  The simulator will stop when this
  value is reached, or when 30 blocks have run.
  \emph{Skip}\index{skip period} specifies the time value of the
  initial skip period.
  Statistics\index{statistics}\index{simulation!statistics} gathered
  during the skip period are discarded.  By default, its value is 0.
  When the run completes, the results reported will be the average
  value of the data collected in all of the blocks.  If the \flag{R}{}
  flag is used, the confidence intervals\index{confidence intervals}
  will for the raw statistics will be included in the monitor
  file\index{file!monitor}\index{monitor file}.
\item[\flag{B}{}, \longopt{blocks}=\emph{blocks[,run-time[,skip]]}]~\\
  Use manual blocking\index{manual blocking}\index{block!manual} with
  \emph{blocks} blocks.  The value of \emph{blocks} must be less than
  or equal to 30.  The run time for each block is specified with
  \emph{run-time}\index{run time!simulation}.  \emph{Skip}\index{skip}
  specifies the time value of the initial skip period.
\item[\flag{C}{}, \longopt{confidence}=\emph{precision[,initial-loops[,run-time]]}]~\\
  Use automatic
  blocking\index{automatic blocking}\index{block!automatic}, stopping
  when the specified precision\index{precision!simulation} is met.
  The run time of each block is estimated, based on
  \emph{initial-loops}\index{initial-loops} running on each
  reference\index{task!reference}\index{reference task} task.  The
  default value for \emph{initial-loops} is 500.  The \emph{run-time}
  argument specifies the maximum total run time.
\item[\flag{d}{}, \longopt{debug}]~\\ This option is used to dump task and entry
  information showing internal index numbers.  This option is useful
  for determining the names of the servers and tasks when tracing the
  execution of the simulator since the Parasol output routines do no
  emit this information at present.  Output is directed to stdout
  unless redirected using \flag{m}{file}.
\item[\flag{e}, \longopt{error}=\emph{error}]~\\
  This option is to enable floating point
  exception\index{floating point!exception} handling.
  \begin{description}
  \item[a] Abort immediately on a floating point error (provided the
    floating point unit can do so).
  \item[b] Abort on floating point errors. (default)
  \item[i] Ignore floating point errors.
  \item[w] Warn on floating point errors.
  \end{description}
  The solver checks for floating point overflow\index{overflow},
  division by zero and invalid operations.  Underflow and inexact
  result exceptions are always ignored.
  
  In some instances,
  infinities\index{infinity}\index{floating point!infinity} will be
  propogated within the solver.  Please refer to the
  \pragma{stop-on-message-loss} pragma below.
\item[\flag{h}{output}]~\\
  Generate comma separated values for the service time distribution\index{service time!distribution}
  data\index{distribution!service time}.  If \emph{output} is a directory, the output file name will be the
  name of a the input file with a \texttt{.csv}\index{output!csv} extension.  Otherwise, the output will be
  written to the named file.
\item[\flag{m}{file}]~\\
  Direct all output generated by the various
  debugging\index{file!debug} and tracing\index{file!tracing} options
  to the monitor\index{file!monitor} file \emph{file}, rather than to
  standard output.  A filename of `\texttt{-}' directs output to
  standard output.
\item[\flag{n}{}, \longopt{no-execute}]~\\
  Read input, but do not solve.  The input is checked for validity. No
  output is generated.
\item[\flag{o}{}, \longopt{output}=\emph{output}]~\\
  Direct analysis results to output\index{output}.  A file name of `\texttt{-}' directs output to standard
  output.  If \emph{output} is a directory, all output from the simulator will be placed there with
  filenames based on the name of the input files processed.  Otherwise, only one input file can be
  processed; its output will be placed in \emph{output}.
\item[\flag{p}{}, \longopt{parseable}]~\\
  Generate parseable output\index{output!parseable} suitable as input
  to other programs such as \manpage{MultiSRVN}{1} and
  \manpage{srvndiff}{1}.  If input is from \texttt{filename},
  parseable output is directed to \texttt{filename.p}.  If standard
  input\index{standard input} is used for input, then the parseable
  output is sent to the standard output device.  If the
  \flag{o}{output} option is used, the parseable output is sent to the
  file name output.  (In this case, only parseable output is emitted.)
\item[\flag{P}, \longopt{pragma}=\emph{pragma}]~\\
  Change the default solution strategy.  Refer to the PRAGMAS chapter
  (\S\ref{sec:lqsim-pragmas})\index{pragma} below for more information.
\item[\flag{R}{}, \longopt{raw-statistics}]~\\
  Print the values of the statistical
  counters\index{counters!statistical}\index{statistical counters} to
  the monitor file\index{file!monitor}.  If the \flag{A}{}, \flag{B}{}
  or \flag{C}{} option was used, the mean value, 95th and 99th
  percentile are reported.  At present, statistics are gathered for
  the task\index{cycle-time!task} and entry\index{cycle-time!entry},
  cycle time task\index{utilization!task},
  processor\index{utilization!processor} and
  entry\index{utilization!entry} utilization, and waiting
  time\index{waiting time} for messages.
\item[\flag{S}{}, \longopt{seed}=\emph{seed}]~\\
  Set the initial seed\index{seed} value for the random
  number\index{random number!generation} generator.  By default, the
  system time from time \manpage{time}{3} is used.  The same seed
  value is used to initialize the random number generator for each
  file when multiple input files are specified.
\item[\flag{t}{}, \longopt{trace}=\emph{traceopts}]~\\
  This option is used to set tracing\index{tracing} options which are
  used to print out various steps of the simulation while it is
  executing.  \emph{Traceopts} is any combination of the following:
  \begin{description}
  \item[\optarg{driver}{}] Print out the underlying tracing
    information from the Parasol\index{Parasol} simulation engine.
  \item[\optarg{processor}{=regex}] Trace activity for
    processors\index{processor!trace}\index{trace!processor} whose
    name match \emph{regex}.  If \emph{regex}is not specified,
    activity on all processors is reported.  \emph{Regex} is regular
    expression of the type accepted by \manpage{egrep}{1}.
  \item[\optarg{task}{=regex}] Trace activity for
    tasks\index{task!trace}\index{trace!task} whose name match
    \emph{regex}.  If \emph{regex} is not specified, activity on all
    tasks is reported.  pattern is regular expression of the type
    accepted by \manpage{egrep}{1}.
  \item[\optarg{events}{regex[:regex]}] Display only events matching
    pattern.  The events are: msg-async, msg-send, msg-receive,
    msg-reply, msg-done, msg-abort, msg-forward, worker-dispatch,
    worker- idle, task-created, task-ready, task-running,
    task-computing, task-waiting, thread-start, thread-enqueue,
    thread-dequeue, thread-idle, thread-create, thread-reap,
    thread-stop, activity-start, activity-execute, activity-fork, and
    activity-join.
  \item[\optarg{msgbuf}{}] Show msgbuf allocation and deallocation.
  \item[\optarg{timeline}{}] Generate events for the timeline tool.
  \end{description}
\item[\flag{T}{}, \longopt{run-time}=\emph{run-time}]~\\
  Set the run time\index{run time!simulation} for the simulation.  The
  default is 10,000 units.  Specifying \flag{T}{} after either
  \flag{A}{} or \flag{B}{} changes the simulation block size, but does
  not turn off blocked statistics
  collection\index{simulation!block}\index{statistics!simulation}.
\item[\flag{v}{}, \longopt{verbose}]~\\
  Print out statistics\index{solution!statistics}\index{statistics}
  about the solution on the standard output device.
\item[\flag{V}{}, \longopt{version}]~\\
  Print out version\index{version} and copyright\index{copyright}
  information.
\item[\flag{w}{}, \longopt{no-warnings}]~\\
Ignore warnings.  The default is to print out all warnings.\index{warning!ignore}
\item[\flag{x}{}, \longopt{xml}]~\\
Generate XML output regardless of input format.
\item[\flag{z}{specialopts}]~\\
  This flag is used to select special options.  Arguments of the form
  $n$ are integers while arguments of the form $n.n$ are real
  numbers.  \emph{Specialopts} is any combination of the following:
  \begin{description}
  \item[\optarg{print-interval}{=nn}] Set the printing interval to
    $n$.  Results are printed after $nn$ blocks have run.  The default
    value is 10.
  \item[\optarg{global-delay}{=n.n}] Set the interprocessor
    delay\index{delay!interprocessor} to nn.n for all tasks.  Delays
    specified in the input file will override the global value.
  \end{description}
\item[\longopt{global-delay}]~\\
Set the inter-processor communication delay to n.n.
\item[\longopt{print-interval}]~\\
Ouptut results after n iterations.
\item[\longopt{restart}]~\\
Re-run the LQX\index{LQX} program without re-solving the models unless a valid solution does not exist.  
This option is useful if LQX print statements are changed, or if a subset of simulations has to be re-run.
\item[\longopt{debug-lqx}]~\\
Output debugging informtion as an LQX\index{LQX!debug} program is being parsed.
\item[\longopt{debug-xml}]~\\
Output XML\index{XML!debug} elements and attributes as they are being parsed.   Since the XML parser usually stops when it encounters an error,
this option can be used to localize the error.
\end{description}

\section{Return Status}
\label{sec:lqsim-return-status}

Lqsim exits 0 on success\index{exit!success}, 1 if the simulation
failed to meet the convergence criteria\index{convergence!failure}, 2
if the input was invalid\index{input!invalid}, 4 if a command line
argument was incorrect\index{command line!incorrect}, 8 for file
read/write problems and -1 for fatal errors\index{error!fatal}.  If
multiple input files\index{input!multiple} are being processed, the
exit code is the bit-wise OR of the above conditions.

\section{Pragmas}
\label{sec:lqsim-pragmas}

Pragmas\index{pragma} are used to alter the behaviour of the simulator
in a variety of ways.  They can be specified in the input file with
``\#pragma'', on the command line with the \flag{P}{} option, or
through the environment variable\index{environment variable}
\texttt{LQSIM\_PRAGMAS}\index{LQSIM\_PRAGMAS@\texttt{LQSIM\_PRAGMAS}}.
Command line\index{command line} specification of pragmas overrides
those defined in the environment variable\index{environment
  variable!override} which in turn override those defined in the input
file.

The following pragmas are supported.  An invalid pragma
specification\index{pragma!invalid!command line} at the command line
will stop the solver.  Invalid pragmas defined in the environment
variable or in the input\index{pragma!invalid!input file} file are
ignored as they might be used by other solvers.
\begin{description}
\item[\optarg{scheduling}{=enum}]~\\
  This pragma is used to select the scheduler used for
  processors\index{processor!scheduling}.  \emph{Enum} is any one of
  the following:
  \begin{description}
  \item[default] Use the scheduler built into parasol for processor
    scheduling.  (faster)
  \item[custom] Use the custom
    scheduler\index{processor!scheduling!custom} for scheduling which
    permits more statistics to be gathered about processor
    utilization\index{utilization!processor}\index{processor!utilization}
    and waiting
    times\index{waiting!processor}\index{processor!waiting}.  However,
    this option invokes more internal tasks, so simulations are slower
    than when using the default scheduler.
  \item[default-natural] Use the parasol scheduler, but don't
    reschedule\index{processor!scheduling!natural} after the end of
    each phase\index{reschedule!phase}\index{phase!reschedule} or
    activity\index{reschedule!activity}\index{activity!reschedule}.
    This action more closely resembles the scheduling of real
    applications.
  \item[custom-natural] Use the custom scheduler; don't reschedule
    after the end of each phase or activity.
  \end{description}
\item[\optarg{messages}{\emph{=n}}]~\\
  Set the number of message buffers\index{message!buffers} to $n$.
  The default is 1000.
\item[\optarg{stop-on-message-loss}{\emph{=bool}}] ~\\
  This pragma is used to control the operation of the solver when the
  arrival rate\index{arrival rate} exceeds the service rate of a
  server.  The simulator can either discard the arrival, or it can
  halt.  The meanings of \emph{bool} are:
  \begin{description}
  \item[false] Ignore queue overflows for open
    arrivals\index{open arrival!overflow} and
    send-no-reply\index{send-no-reply!overflow} requests.  If a queue
    overflows\index{overflow}, its waiting times is reported as
    infinite\index{infinity}.
  \item[true] Stop if messages are lost.
  \end{description}
\item[\optarg{reschedule-on-async-send}{\emph{=bool}}]~\\
  In models with send-no-reply\index{send-no-reply} messages, the
  simulator does not reschedule the processor after an asynchronous
  message is sent (unlike the case with synchronous messages).  The
  meanings of \emph{bool} are:
  \begin{description}
  \item[true] reschedule after each asynchronous message.
  \item[false] reschedule after each asynchronous message.
  \end{description}
\end{description}

\section{Stopping Criteria}
\label{sec:stopping-criteria}\index{stopping criteria}

It is important that the length of the
simulation\index{length!simulation} be chosen properly.  Results may
be inaccurate if the simulation run is too short.  Simulations that
run too long waste time and resources.

Lqsim uses \emph{batch means}\index{batch means} (or the method of
samples\index{method of samples}) to generate confidence
intervals\index{confidence intervals}.  With automatic
blocking\index{automatic blocking}\index{block!automatic}, the
confidence intervals are computed after the simulations runs for three
blocks plus the initial skip period\index{skip period} If the root or
the mean of the squares\index{root mean square} of the confidence
intervals for the entry service times\index{service time!entry} is
within the specified precision, the simulation stops.  Otherwise, the
simulation runs for another block and repeats the test.  With manual
blocking\index{block!manual}, lqsim runs the number of blocks
specified then stops.  In either case, the simulator will stop after
30 blocks.

Confidence intervals can be tightened by either running additional
blocks or by increasing the block size\index{block!size}.  A rule of
thumb is the block size should be 10,000 times larger than the largest
service time demand\index{service time!demand} in the input model.

\section{Model Limits}
\label{sec:lqsim-model-limits}

The following table lists the acceptable parameter types and limits
for lqsim.  An error will be reported if an unsupported parameter is
supplied except when the value supplied is the same as the default.

\begin{table}[htbp]
  \centering
  \begin{tabular}[c]{ll}
    Parameter&lqsim \\
    \hline
    Phases\index{phase!maximum} & 3 \\
    Scheduling\index{scheduling} & FIFO, HOL, PPR, RAND \\
    Open arrivals\index{open arrival} & yes \\
    Phase type\index{phase!type} & stochastic, deterministic \\
    Think Time\index{think time}  & yes  \\
    Coefficient of variation\index{coefficient of variation} & yes \\
    Interprocessor-delay\index{interprocessor delay} & yes \\
    Asynchronous connections\index{asynchronous connections} & yes \\
    Forwarding\index{forwarding} & yes \\
    Multi-servers\index{multi-server} & yes \\
    Infinite-servers\index{infinite server} & yes \\
    Max Entries\index{entry!maximum} & unlimited \\
    Max Tasks\index{task!maximum} & 1000 \\
    Max Processors\index{processor!maximum} & 1000 \\
    Max Entries per Task & unlimited \\
  \end{tabular}
  \caption{\label{tab:lqsim-model-limits}Lqsim Model Limits\index{limits!lqsim}}
\end{table}

%%% Local Variables: 
%%% mode: latex
%%% mode: outline-minor 
%%% fill-column: 108
%%% TeX-master: "userman"
%%% End: 
