\C 
\C $Id$
\C 
\C $Log$
\C Revision 31.0  2005/01/02 02:30:29  greg
\C Bring all files up to revision 31.
\C
\C Revision 30.0  2003/11/12 18:56:52  greg
\C Bring all files up to revision 30.0
\C
\C Revision 29.0  2003/06/13 00:48:38  greg
\C bounce version number.
\C
\C Revision 1.4  1995/01/11 21:05:12  greg
\C Revised for R12 of MVA software.
\C
\C Revision 1.3  1994/08/04  20:52:11  greg
\C Intermediate save
\C
\C Revision 1.2  1994/06/24  19:11:32  greg
\C Format changes.
\C
\C Revision 1.1  1994/06/23  19:11:01  greg
\C Initial revision
\C
\C ----------------------------------------------------------------------
\section{MVA}
\label{sec:mva}

The MVA class is used to solve queueing network models.  Typically,
only one instance will exist at any instant, although there is no
restriction on the number of instances of this class.

The class relationships for the MVA server module are shown in
\htmlonly{the \link{figure below}{fig:mva}}\texonly{
  Figure~\ref{fig:mva}}.  The class \file{MVA} is an abstract
superclasses which is not used directly.

\begin{figure}[htbp]
  \label{fig:mva}
  \begin{center}
    \T \tex \leavevmode \C 
\C $Id$
\C 
\C $Log$
\C Revision 31.0  2005/01/02 02:30:29  greg
\C Bring all files up to revision 31.
\C
\C Revision 30.0  2003/11/12 18:56:52  greg
\C Bring all files up to revision 30.0
\C
\C Revision 29.0  2003/06/13 00:48:38  greg
\C bounce version number.
\C
\C Revision 1.4  1995/01/11 21:05:12  greg
\C Revised for R12 of MVA software.
\C
\C Revision 1.3  1994/08/04  20:52:11  greg
\C Intermediate save
\C
\C Revision 1.2  1994/06/24  19:11:32  greg
\C Format changes.
\C
\C Revision 1.1  1994/06/23  19:11:01  greg
\C Initial revision
\C
\C ----------------------------------------------------------------------
\section{MVA}
\label{sec:mva}

The MVA class is used to solve queueing network models.  Typically,
only one instance will exist at any instant, although there is no
restriction on the number of instances of this class.

The class relationships for the MVA server module are shown in
\htmlonly{the \link{figure below}{fig:mva}}\texonly{
  Figure~\ref{fig:mva}}.  The class \file{MVA} is an abstract
superclasses which is not used directly.

\begin{figure}[htbp]
  \label{fig:mva}
  \begin{center}
    \T \tex \leavevmode \C 
\C $Id$
\C 
\C $Log$
\C Revision 31.0  2005/01/02 02:30:29  greg
\C Bring all files up to revision 31.
\C
\C Revision 30.0  2003/11/12 18:56:52  greg
\C Bring all files up to revision 30.0
\C
\C Revision 29.0  2003/06/13 00:48:38  greg
\C bounce version number.
\C
\C Revision 1.4  1995/01/11 21:05:12  greg
\C Revised for R12 of MVA software.
\C
\C Revision 1.3  1994/08/04  20:52:11  greg
\C Intermediate save
\C
\C Revision 1.2  1994/06/24  19:11:32  greg
\C Format changes.
\C
\C Revision 1.1  1994/06/23  19:11:01  greg
\C Initial revision
\C
\C ----------------------------------------------------------------------
\section{MVA}
\label{sec:mva}

The MVA class is used to solve queueing network models.  Typically,
only one instance will exist at any instant, although there is no
restriction on the number of instances of this class.

The class relationships for the MVA server module are shown in
\htmlonly{the \link{figure below}{fig:mva}}\texonly{
  Figure~\ref{fig:mva}}.  The class \file{MVA} is an abstract
superclasses which is not used directly.

\begin{figure}[htbp]
  \label{fig:mva}
  \begin{center}
    \T \tex \leavevmode \C 
\C $Id$
\C 
\C $Log$
\C Revision 31.0  2005/01/02 02:30:29  greg
\C Bring all files up to revision 31.
\C
\C Revision 30.0  2003/11/12 18:56:52  greg
\C Bring all files up to revision 30.0
\C
\C Revision 29.0  2003/06/13 00:48:38  greg
\C bounce version number.
\C
\C Revision 1.4  1995/01/11 21:05:12  greg
\C Revised for R12 of MVA software.
\C
\C Revision 1.3  1994/08/04  20:52:11  greg
\C Intermediate save
\C
\C Revision 1.2  1994/06/24  19:11:32  greg
\C Format changes.
\C
\C Revision 1.1  1994/06/23  19:11:01  greg
\C Initial revision
\C
\C ----------------------------------------------------------------------
\section{MVA}
\label{sec:mva}

The MVA class is used to solve queueing network models.  Typically,
only one instance will exist at any instant, although there is no
restriction on the number of instances of this class.

The class relationships for the MVA server module are shown in
\htmlonly{the \link{figure below}{fig:mva}}\texonly{
  Figure~\ref{fig:mva}}.  The class \file{MVA} is an abstract
superclasses which is not used directly.

\begin{figure}[htbp]
  \label{fig:mva}
  \begin{center}
    \T \tex \leavevmode \input{mva.epic}
    \caption{MVA class hierarchy}
    \H \htmlimage{mva.gif}
  \end{center}
\end{figure}

This class relies on \link{\file{Server}}[ (\Sec\Ref)]{sec:server} for the waiting
time calculations for each station.

\htmlrule
\subsection{Important Performance Notes}
\label{sec:mva-cautions}

Common expression hoisting has been performed to reduce computational
costs.  The expressions are marked with the comment ``\code{Hoist}''
in the source code.  It is assumed that the size and the maximum
number of customers for all \link{\file{Population}}{sec:population}
vectors for all stations is the same so that the offset for a given
population value \var{N} is the same for all objects of type
\file{Population}.  The offset computation is hoisted out of the loops
where possible.  The integer variable \var{n} represents the array
offset for the customer population \var{N}, i.e., \code{n =
  population.offset(N)}.  The integer variable \var{Nej} represents
the array offset of customer population \var{N} with one customer of
class \emph{j} removed.  

The functions in class
\link{\file{Population}}[~(\Sec\Ref)]{sec:population} exported to aid
hoisting are:
\begin{itemize}
\item \code{unsigned offset( const PopVector & N ) const;}
\item \code{unsigned offset_e_j( const PopVector & N, const unsigned j );}
\item \code{unsigned offset_e_c_e_j( const unsigned c, const unsigned j);}
\end{itemize}

The instance variable \link{\var{c}}{sec:mva-ivars} used in the
\link{Linearizer solver}[ (\Sec\Ref)]{sec:mva-linearizer} and subclasses
is also very special.  It represents the class of the customer being
removed by the linearizer algorithm.  This variable is not be
explicitly visible.  However, it is used, so \bold{DO NOT USE} \var{c}
as a local variable \bold{ANYWHERE}.  It is used in \bold{ALL CLASSES}
including MVA and especially \code{MVA::step()}.

\subsection{Class MVA --- Abstract Superclass}

\subsubsection{Instance Variables}
\label{sec:mva-ivars}

\begin{verbatim}
public:
    static int boundsLimit;     /* Enable bounds limiting.      */
    Population ***L;            /* Queue length.                */
    Population ***U;            /* Station utilization.         */
protected:
    const PopVector NCust;      /* Number of Customers.         */
    const unsigned M;           /* Number of stations.          */
    const unsigned K;           /* Number of classes.           */
    Server **Q;                 /* Queue type.  SS/delay.       */
    unsigned c;                 /* Cust. from class c removed.  */
    Population **P;             /* For marginal probabilities.  */
private:
    Vector<double> X;           /* Throughput per class.        */
    Vector<double> Z;           /* Think time per class.        */
    Vector<Interlock> IL;       /* Interlock information.       */
    unsigned long coreCount;    /* Number of iterations of step */
\end{verbatim}


Access to all instance variables for the MVA class is restricted to
members of the class.  
\begin{description}
\item[c] \texonly{---} Class of customer removed by
  \link{Linearizer}{sec:mva-linearizer}.  Not used by exact MVA and
  Schweitzer solvers, but defined here to allow for common expressions
  in the MVA \link{\code{step()}}{sec:mva-step} function.
\item[coreCount] \texonly{---} A counter that is incremented each time
  the \link{\code{step()}}{sec:mva-step} function is executed.
\item[K] \texonly{---} The maximum number of \bold{classes} at any station.
\item[L] \texonly{---} Queue \underline{L}ength.  Dimensioned by \bold{entry} and \bold{class}.
\item[M] \texonly{---} The number of \bold{stations} in the network.
\item[NCust] \texonly{---} A \link{vector}{sec:vector} containing the number of
  customers for each \bold{class}.  Classes range from 1 to \var{K}.
  \var{NCust[0]} is not used.
\item[P] \texonly{---} The \link{marginal
    probabilities}{sec:mva-marginal} for each station in the network.
  Population space is allocated only for those stations that need
  marginal probabilities (for example, multiservers).  \bold{Note:}
  \var{P[m][J][N]} is \var{PB[m][N]} where \var{J} is the number of
  servers at a multiserver.
\item[Q] \texonly{---} The \link{stations}{sec:server} in the network.  It is an
  array of pointers to stations and is indexed from 1 to \var{M}.
  \var{Q[0]} is not used.  
\item[U] \texonly{---} Station \underline{U}tilization.  Dimensioned by \bold{entry}
  and \bold{class}. 
\item[X] \texonly{---} the throughput for each \bold{class}.  It is
  valid for the last \link{population}{sec:population} vector solved
  (which is usually the maximum customer population.)
\item[Z] \texonly{---} The think time for each \bold{class}.
each time the MVA core step is executed.
\end{description}

\subsubsection{Variable Conventions}
The following variables are typically used throughout the MVA classes
either as arguments to, or locally within, a member function.

\label{sec:mva-conventions}
\begin{description}
\item[E] \texonly{---} The total number of entries for a station.
\item[e] \texonly{---} An entry index: \math[1 <= e <= E]{1 \leq e \leq E}.
\item[J] \texonly{---} The total number of servers at a multiserver
  station.
\item[j]  \texonly{---} A server index for multiservers: : \math[1 <=
  j <= J]{1 \leq j \leq J}.
\item[k] \texonly{---} A class (chain) index: \math[1 <= k <= K]{1
    \leq K \leq K}.
\item[m] \texonly{---} A station index: \math[1 <= m <= M]{1 \leq m
    \leq M}.
\item[N] \texonly{---} A Population \link{vector}{sec:vector}.
  Objects of type \link{\file{Population}}{sec:population} are indexed
  by this type.
\item[n] \texonly{---} A \emph{precomputed} index into the array that
  stores values for a given population \var{N}.
\end{description}


\subsubsection{Constructors for class MVA}

All solvers are created with the following type of call.  Subclasses
are expected to accept the same argument list.  The argument \var{m}
specifies the number of stations in the network.  \var{k} specifies
the number of classes.  \var{q} an array of pointers to objects of
type Server.  These objects are the stations in the queueing network.
Last, the vector \var{N} defines the maximum number of customers in
each class \var{k}.  Index 0 (zero) is NOT used in either \var{q} or
\var{N}.  Range checking does not take place for array \var{q}.

\begin{verbatim}
MVA( unsigned m, unsigned k, Server **q, Vector& N );
\end{verbatim}

\subsubsection{Instance Variable Access}

\begin{description}
\item[thinkTime] \texonly{---} Think Time.\\
  \code{double& thinkTime( const unsigned k );}

  The instance variable \var{Z}, (think time) can be set or retrieved
  with this function.
\end{description}

\subsubsection{Queries}

\begin{description}
\item[throughput] \texonly{---} Throughput.\\
  \code{double throughput( const unsigned e, const unsigned k ) const;}\\
  \code{double throughput( const unsigned e ) const;}

  Return the station's throughput for a given \bold{entry} and
  [optionally] \bold{class}.

\item[utilization] \texonly{---} Return the utilization of the station.\\
  \code{double utilization( const Server& station, const PopVector& N,
    const unsigned j ) const;} \\
  \code{double utilization( const Server& station, const unsigned k,
    const PopVector& N, const unsigned j ) const;} \\
  \code{double utilization( const unsigned m, const unsigned k, const
    PopVector& N ) const;} 

  The first of the three member functions returns the total
  utilization of \var{station} with one customer removed from class
  \var{j}.  The second member function returns the utilization for
  class \var{k} with one customer removed from class \var{j}.  The
  final version returns the utilization for class \var{k} at the full
  customer population \var{N}.

\item[sumOf_SL_m] \texonly{---} Queue length sum.\\
  \code{double sumOf_SL_m( const Server& s, const unsigned e, const
    PopVector &N, const unsigned j) const;}

  The function \code{sumOf_SL_m()} returns the sum of the service time
  multiplied by the queue length at station \var{s}, entry \var{e}, at
  the \link{population}{sec:population} \var{N} when one customer is
  removed from class \var{j}; i.e., \math[]{\sum_k s_{mk} L_m({\bf
      N}-e_{j})}.  This function is called from the
  \link{\file{Server}}{sec:server} class.  Its implementation depends on the
  MVA solver class.

\item[sumOf_p] \texonly{---} \math[Sum of Pi]{\sum (J_i-1-j)P_i(j,{\bf N} - e_j)}.\\
  \code{double sumOf_p( const Server& station, const PopVector &N, const
    unsigned j ) const}

  Return the \emph{something} with one customer removed from class
  \var{j} for population \var{N}. (See \link{Eqn. (2.19) and
    (3.8)}[~\Cite]{queue:reiser-80}, \link{Eqn.
    (2.1)}[~\Cite]{queue:conway-88}.)

\item[PB] \texonly{---} Pr(All Servers Busy)\\
  \code{double PB( const Server& station, const PopVector &N, const unsigned j ) const;}

  Return the probability that all servers are busy for a population of
  \var{N} and one customer removed from class \var{j}.

  Note: \math[PB(N) = P(J,N)]{{\it PB}_{m}({\bf N}) = P_{m}(J,{\bf N})}

\item[queueLength] \texonly{---} Find queue length.\\ \code{double
    queueLength( const Server&, const unsigned k, const PopVector& N,
    const unsigned j ) const;}

  Find the queue length at the server for class \var{k} with one customer
  from class \var{j} removed from the population \var{N}.  This function
  \emph{does not include} customers in service at the station. 

\item[iterations] \texonly{---} Number of core steps.\\
  \code{unsigned long iterations() const;}

  Returns the number of times the core MVA
  \link{\code{step}}{sec:mva-step} is called.

\item[offset] \texonly{---} Array offset calculation.\\
  \code{virtual unsigned offset( const PopVector& N ) const;} \\
  \code{virtual unsigned offset_e_j( const PopVector& N, const unsigned
    j ) const;} \\

  Calculate and return the \link{internal array
    offset}{sec:population-offset} for the
  \link{population}{sec:population} vector \var{N}.  These functions are
  defined by subclasses so that the proper offset function in the
  \link{\file{Population}}{sec:population} class can be called.  (Exact
  MVA uses \link{general populations}{sec:general-population} while the
  other solvers use \link{special populations}{sec:special-population}).

\end{description}

\subsubsection{Computation}
\begin{description}
\label{sec:mva-solve}
\item[solve] \texonly{---} Solve the model\\
\code{void solve();}

\code{Solve()} is used to solve the queueing network.  It invokes the
fuction \link{\code{step()}}{sec:mva-step} to perform the actual MVA
calculation.  Implemented by subclasses.


\label{sec:mva-step}
\item[step] \texonly{---} Core MVA step.\\
\code{void step( const PopVector& N );}

The \code{step()} member function is used to compute the queue lenght
and untilization at each station in the queueing network for the
population \var{N}.  It assumes that solution for the network exists
with one customer removed from each class \var{k}.  It first finds the
waiting time at each station (See Eqn.
\link{(1)}[~\Cite]{queue:chandy-82}). Next, the throughputs for each
class, \math[\var{X[k]}]{X_k}, are found.  This result is then used to
find the new queue length \math[\var{L}]{L_{mk}({\bf N})} (See Eqn.
\link{(6)}[~\Cite]{queue:chandy-82})

\begin{iftex}
\tex
\begin{equation}
  L_{mk}({\bf N}) = N_k \frac{ v_{mk} W_{mk}({\bf N}) }{
    \sum_{m^\prime} v_{m^\prime k} W_{m^\prime k}({\bf N})}
\end{equation}
\end{iftex}

The waiting time \math[\var{W}]{W_{mk}} is calculated by the class
\link{\code{Server}}[~(\Sec\Ref)]{sec:server}.  It supplies two
member functions, \link{\code{initStep()}}{sec:server-initstep} and
\code{wait()}.  \code{InitStep()} is called first and is used to
initialize any data prior to finding the new waiting time.
\code{Wait()} is used to find the waiting time at the current
population value.  Both functions are overridden by subclasses of
\file{Server} to provide the necessary functionality.
\begin{iftex}
  For example, Equation~\ref{eqn:delay} is used to find the waiting
  time for delay servers, Equation~\ref{eqn:fcfs} is used for First
  Come, First Served servers, and Equation~\ref{eqn:msfcfs} is used
  for multi-servers.
\tex
\begin{eqnarray}
W_{mk}({\bf N}) & = & s_{mk} \label{eqn:delay} \\
W_{mk}({\bf N}) & = & s_{mk} [ 1 + L_{mk}({\bf N} - e_k) ]
\label{eqn:fcfs} \\
W_{mk}({\bf N}) & = & \frac{ s_{mk} }{J_{m}} \left[ 1 + L_{mk}({\bf N} -
e_k) + \sum_{j=0}^{J_{m} - 2} (J_{m} - 1 - j) P(j,{\bf N} - e_k) \right]
\label{eqn:msfcfs} 
\end{eqnarray}
\end{iftex}

Note that \code{step()} may be called multiple times for a given
population (in particular, by the approximate solvers), while
\code{initStep()} is called only once.

\item[utilization] \texonly{---} Find utilization\\
\code{double utilization( const unsigned m, const unsigned e, const
  unsigned k, const double L ) const;}

Compute the utilization \math[\var{U}]{U_{mk}} for the various
stations and customer populations.  (See \link{Eqn.
  (7)}[~\Cite]{queue:chandy-82})

\begin{iftex}
\tex
\begin{equation}
  U_{mk}({\bf N}) = \frac{s_{mk} L_{mk}({\bf N})}{W_{m^\prime k}({\bf N})}
\end{equation}
\end{iftex}

\item[marginalProbabilities] \texonly{---} Compute P.\\
\code{void marginalProbabilities( const unsigned m, const PopVector& N );}

Compute the marginal probability that there are \var{j} customers at
station \var{m} for population \var{N}. The probabilities are stored
in \link{\var{P[m][j][N]}}{sec:mva-ivars}.  \bold{Note:} The
probability that all servers are busy is denoted as \var{PB} in the
literature, but is stored in \var{P} in the expected place (i.e.
\math[\var{j} = \var{M}]{j = J}).  (See Equations \link{(2.10) and
  (2.13) and (3.9)}[~\Cite]{queue:reiser-80} and \link{(2.4), (2.5)
  and (2.6)}[~\Cite]{queue:conway-88}).

\begin{iftex}
\tex
\begin{eqnarray}
P_{m}(j,{\bf N}) & = & \sum_{k=1}^{K} s_{mk}\lambda_{mk}({\bf N}))
P_{m}(j-1,{\bf N} - e_k) / j, \forall j, 0 < j < J \\
P_{m}(J,{\bf N}) & = & \sum_{k=1}^{K} s_{mk}\lambda_{mk}({\bf N}))
\left[ P_{m}(J,{\bf N} - e_k) + P_{m}(J-1,{\bf N} - e_k) \right] / J \\
P_{m}(0,{\bf N}) & = & 1 - \sum_{j=1}^{J} P_{m}(j,\bf{N})
\end{eqnarray}

Note that $P_{m}(J,{\bf N}) = PB_m({\bf N})$.
\end{iftex}

\end{description}

\subsubsection{Printing}
\begin{description}
\item[print] \texonly{---} Print all results.\\ \code{ostream& print(
    ostream& os );}

  Print the throughput and utilization for all
  \link{stations}{sec:server} using the
  \link{\code{printResults}}{sec:server-print} function.

\item[printL] \texonly{---} Print Queue Lengths.\\ \code{ostream&
    printL( ostream& os, const PopVector& N );}

  Print the value of queue length \link{\var{L}}{sec:server-ivars} at
  the \link{population}{sec:population} \var{N}.  This function is
  used to debug the solvers.

\item[printP] \texonly{---} Print marginal probabilities.\\ 
  \code{ostream& printZ( ostream& os );}

  Print marginal probabilities for all classes.

\item[printW] \texonly{---} Print waiting times.\\ \code{ostream&
    printL( ostream& os );}

  Print the value of waiting time \link{\var{W}}{sec:server-ivars}.
  This function is used to debug the solvers.

\item[printX] \texonly{---} Print throughputs.\\ 
  \code{ostream& printZ( ostream& os );}

  Print throughputs for all classes.

\item[printZ] \texonly{---} Print think times.\\ 
  \code{ostream& printZ( ostream& os );}

  Print think times for all classes.

\end{description}


\subsection{Class ExactMVA}
\label{sec:mva-exact}

The Exact MVA algorithm is taken from \link{Chandy and
  Neuse}[~\Cite]{queue:chandy-82} (Also refer to \link{Reiser and
  Lavenberg}[~\Cite]{queue:reiser-80} and \link{Lazowska et.
  al.}[~\Cite]{perf:lazowska:84}).  It sequences through all customer
populations, starting at \math[0,0,...0]{0_1,0_2,...,0_K}, calling the
\link{\code{step()}}{sec:mva-step} function for each.

\subsubsection{Queries}
\begin{description}
\item[offset] \texonly{---} Return array offset for population\\ 
  \code{unsigned offset( const PopVector& N ) const}\\ \code{unsigned
    offset_e_j( const PopVector& N, const unsigned j ) const}

  Return \link{population}{sec:population} array offset.

\end{description}
\subsubsection{Computation}
\begin{description}
\item[solve] \texonly{---} Solve model\\ \code{void solve()};

  Recursively solve for population vector \var{N} starting at a
  population of \math[0,0,...,0]{[0_1,0_2,...0_K]} to \var{nCust}.
  Solutions for each \link{population}{sec:population} vector in the
  population space are generated by calling \code{nextPop()}.  The
  dimensionality of \var{N} is limited by stack size.

Note: while is is possible to save space by not retaining
\link{population}{sec:population} values for the ``outermost'' class
(see \link{Lazowska}[~\Cite]{perf:lazowska:84}, Chapter 19), no
attempt is done so here.

\end{description}

\subsection{Class Schweitzer}
\label{sec:mva-schweitzer}

\subsubsection{Instance Variables}
\label{sec:mva-schweitzer-ivars}
\begin{verbatim}
private:
    double ***last_L;               /* For local comparison.        */
\end{verbatim}

\begin{description}
\item[last_L] \texonly{---} Previous value of the queue length \var{L}
  and is used to test for the termination condition.
\end{description}

\subsubsection{Initialization}
\begin{description}
\label{sec:mva-schweitzer-init-L}
\item[init_L] \texonly{---} Initialize Queue Lengths\\
\code{virtual void init_L( const PopVector & N);}

Find the queue length based on fraction \var{F} of class \var{k} jobs
at station for population \var{N} (See \link{Eqn.
  (11)}[~\Cite]{queue:chandy-82}).

\begin{iftex}
\tex
\begin{equation}
L_{mk}({\bf N} - e_j) = ({\bf N} - e_j)F_{mk}({\bf N}) + D_{mk}(\bf N)
\end{equation}
\end{iftex}

\item[init_U] \texonly{---} Initialize Utilizations\\
\code{virtual void init_U( const PopVector & );}

Initialize the utilization array \math[\var{U}]{U \forall m,e,k} for
population \math[\var{N} with one customer removed from each class in
turn]{N - e_j, 1 \leq j \leq K}.  \code{init_U()} is separated from
\link{\code{init_L()}}{sec:mva-schweitzer-init-L} to permit subclasses to override.

\end{description}
        
\subsubsection{Queries}
\begin{description}
\item[D_mekj] \texonly{---} Return D\\
\code{virtual double D_mekj(unsigned m,unsigned e,unsigned k,unsigned j);}

Return the change in the fraction of class \var{k} jobs at queue
\var{m} after one class \var{j} job has been removed.  For the
Schweitzer solver, the best estimate of \var{D} is zero.

\end{description}
\subsubsection{Computation}

\begin{description}
\item[solve] \texonly{---} Solve model using Schweitzer method.\\
\code{virtual void solve();}

\code{solve()} merely is a public interface to \code{core()}.

\label{sec:mva-schweitzer-core}
\item[core] \texonly{---} \\
\code{void core( const PopVector & );}

\code{Core()} is the basic iterative solver for Bard Schweitzer and
Linearizer algorithms.  For each iteraction, it first calls
\link{\code{init_L()}}{sec:mva-schweitzer-init-L} and \code{init_U()}
followed by \link{\code{step()}}{sec:mva-step}.

\end{description}

\subsection{Class Linearizer}
\label{sec:mva-linearizer}

The \link{linearizer
  algorithm}[~\Cite]{queue:chandy-82,queue:krzesinski-84,queue:conway-88}
improves upon the \link{Schweitzer}{sec:mva-schweitzer} method by
providing better estimates for the variable \math[\var{D}]{D_{mejk}}.
Linearizer calls the Schweitzer solver \math[K+1]{K+1} times per outer
loop: once each with one customer from class \math[\var{c}]{c}
removed, and once at the maximum customer population.  The outer loop
is executed three times.

\subsubsection{Instance Variables}
\label{sec:mva-linearizer-ivars}
\begin{verbatim}
protected:
    Population ***F;            /* Fraction of jobs at queue.   */
    Population ***saved_L;      /* Saved queue length.          */
    Population ***saved_U;      /* Saved utilization.           */
    double ****D;               /* Delta Fraction of jobs.      */
\end{verbatim}

\begin{description}
\item[D] \texonly{---} The change in the fraction of queue length for
  station \var{m}, entry \var{e}, class \var{k} when a class \var{j}
  customer is removed.
\item[F] \texonly{---} The fraction of class \var{k} jobs at station
  \var{m}, entry \var{e}. 
\item[saved_L] \texonly{---} Saved queue lengths.
\item[saved_U] \texonly{---} Saved station utilizations.
\end{description}

\subsubsection{Initialization}
\begin{description}
\label{sec:mva-linearizer-init-D}
\item[init_D] \texonly{---} Intialize \var{D}\\
\code{virtual void init_D( const PopVector & N );}

Initialize \var{D}, the difference in queue lengths when a customer
from class \var{j} is removed (See Eqn.
\link{(10)}[~\Cite]{queue:chandy-82}).  This member function is
overridden for the \link{fast linearizer}{sec:mva-fast} solver.
\begin{iftex}
\tex
\begin{equation}
D_{mkj}({\bf N}) = F_{mk}({\bf N} - e_j) - F_{mk}({\bf N})
\end{equation}
\end{iftex}

\end{description}

\subsubsection{Queries}

\begin{description}

\item[D_mekj] \texonly{---} Return D\\
code{double D_mekj( unsigned m, unsigned e, unsigned k, unsigned j );}

Return the change in the fraction of class \var{k} jobs at queue
\var{m} after one class \var{j} job has been removed.  This function
overrides the member function supplied by the \file{Schweitzer} class because
\var{D} is now computed.

\end{description}
        
\subsubsection{Computation}

\begin{description}
\item[solve] \texonly{---} Solve model using Linearizer method.\\
\code{virtual void solve();}

Solve the model using the linearizer solution strategy.  This member
function calls the Schweitzer solver,
\link{\code{core()}}{sec:mva-schweitzer-core} with one customer
removed from each class in turn and at the full population value.  The
instance variable \link{\var{c}}{sec:mva-ivars}, defined in the
abstract superclass, is set according to the class in which the
customer is removed.

The member function \link{\code{init_D()}}{sec:mva-linearizer-init-D}
is called to initialize the \var{D} array before each call to the
Schweitzer \code{core()} solver.  

\end{description}

\subsubsection{Input and Output}
\begin{description}
\item[print] \texonly{---} Printing fuctions\\
\code{ostream& printF( ostream&, const PopVector& N );}\\
\code{ostream& printD( ostream&, const PopVector& N );}

Print out the value of the instance variables \var{F} and \var{D} for
the population \var{N} respecitively.  Used for debugging only.

\end{description}

\subsection{Class Linearizer2}
\label{sec:mva-fast}

Algorithm is from \link{de Souza e Silva and Muntz}[~\Cite]{queue:deSouzaeSilva-90}.

\subsubsection{Instance Variables}
\label{sec:mva-fast-ivars}
\begin{verbatim}
private:
    SpecialPop **Lm;            /* Queue length sum (Fast lin.) */
    SpecialPop **Um;            /* Queue length sum (Fast lin.) */
    double ***D_k;              /* Sum over k.                  */
\end{verbatim}

\begin{description}
\item[D_k] \texonly{---} \math[Sum over class \var{k} of
  \var{D}]{D^\prime_{mj} = \sum_{k=1}^{K} N_k D_{mkj}({\bf N})}.
\item[Lm] \texonly{---} Queue \underline{L}ength summed over all stations.  Used by
  the \link{fast linearizer}{sec:mva-fast}.  Dimensioned by \bold{entry}.
\item[Um] \texonly{---} Station \underline{U}tilization.  Used by the \link{fast
    linearizer}{sec:mva-fast}.  Dimensioned by \bold{entry}. 
\end{description}
\subsubsection{Initialization}

\begin{description}
\label{sec:mva-fast-init-D}
\item[init_D] \texonly{---} Initialize D array.\\
\code{void init_D( const PopVector & );}

Initialize \var{D}, the difference in queue lengths when a customer
from class \var{j} is removed.  This version of the \code{init_D()}
member function also initializes \math[\var{D_k}]{D^\prime_{mj}}.


\label{sec:mva-linearizer-init-L}
\item[init_L] \texonly{---} Initialize Queue lengths.\\
\code{void init_L( const PopVector & );}

Pre-compute \math[\var{S} times \var{Lm}]{S \cdot L_{mk}} and save.
\math[\var{S} times \var{L}]{S \cdot L_{mk}} must be pre-computed
because it relies on the old values of \math[\var{Lm}]{L_{mk}} for all
populations of \math[\var{N}]{\bf N}.  The expressions from \link{de
  Souza e Silva and Muntz}[~\Cite]{queue:deSouzaeSilva-90} (Equations
(8) and (9)) do not multiply through by service times --- this version
does to simplify the expressions for the \link{High Variability First
  Come, First Served}{sec:server-hvfcfs} Server.

\begin{iftex}
\tex
\begin{eqnarray}
L_{m}({\bf N} - e_j ) & = & L_{m}({\bf N}) - \frac{L_{m}({\bf N})}{N_j} +
D^\prime_{mj}({\bf N}) - D_{mjj}({\bf N}) \\
L_{m}({\bf M} - e_j ) & = & L_{m}({\bf M}) - \frac{L_{m}({\bf M})}{M_j} +
D^\prime_{mj}({\bf M}) - D_{mjj}({\bf M}) + - D_{mcj}({\bf M}) \\
{\bf M} & = & {\bf N} - e_c \nonumber
\end{eqnarray}
\end{iftex}

\item[init_U] \texonly{---} Initialize Utilizations.\\
\code{void init_U( const PopVector & );}

Initialize the utilization array \math[\var{U}]{U \forall m,e,k} for
population \math[\var{N}]{N - e_j, 1 \leq j \leq K}.

\end{description}
\subsubsection{Queries}

\begin{description}

\item[sumOf_SL_m] \texonly{---} Queue length sum.\\
\code{double sumOf_SL_m( const Server& s, const unsigned e, const
  PopVector &N, const unsigned j ) const;}

Return the pre-computed value of \var{Lm} for entry \var{e}, class
\var{j}.  This value was calculated in
\link{\code{init_L()}}{sec:mva-linearizer-init-L} earlier.

\end{description}

\C Local Variables: 
\C mode: latex
\C TeX-master: "lqns"
\C End: 

    \caption{MVA class hierarchy}
    \H \htmlimage{mva.gif}
  \end{center}
\end{figure}

This class relies on \link{\file{Server}}[ (\Sec\Ref)]{sec:server} for the waiting
time calculations for each station.

\htmlrule
\subsection{Important Performance Notes}
\label{sec:mva-cautions}

Common expression hoisting has been performed to reduce computational
costs.  The expressions are marked with the comment ``\code{Hoist}''
in the source code.  It is assumed that the size and the maximum
number of customers for all \link{\file{Population}}{sec:population}
vectors for all stations is the same so that the offset for a given
population value \var{N} is the same for all objects of type
\file{Population}.  The offset computation is hoisted out of the loops
where possible.  The integer variable \var{n} represents the array
offset for the customer population \var{N}, i.e., \code{n =
  population.offset(N)}.  The integer variable \var{Nej} represents
the array offset of customer population \var{N} with one customer of
class \emph{j} removed.  

The functions in class
\link{\file{Population}}[~(\Sec\Ref)]{sec:population} exported to aid
hoisting are:
\begin{itemize}
\item \code{unsigned offset( const PopVector & N ) const;}
\item \code{unsigned offset_e_j( const PopVector & N, const unsigned j );}
\item \code{unsigned offset_e_c_e_j( const unsigned c, const unsigned j);}
\end{itemize}

The instance variable \link{\var{c}}{sec:mva-ivars} used in the
\link{Linearizer solver}[ (\Sec\Ref)]{sec:mva-linearizer} and subclasses
is also very special.  It represents the class of the customer being
removed by the linearizer algorithm.  This variable is not be
explicitly visible.  However, it is used, so \bold{DO NOT USE} \var{c}
as a local variable \bold{ANYWHERE}.  It is used in \bold{ALL CLASSES}
including MVA and especially \code{MVA::step()}.

\subsection{Class MVA --- Abstract Superclass}

\subsubsection{Instance Variables}
\label{sec:mva-ivars}

\begin{verbatim}
public:
    static int boundsLimit;     /* Enable bounds limiting.      */
    Population ***L;            /* Queue length.                */
    Population ***U;            /* Station utilization.         */
protected:
    const PopVector NCust;      /* Number of Customers.         */
    const unsigned M;           /* Number of stations.          */
    const unsigned K;           /* Number of classes.           */
    Server **Q;                 /* Queue type.  SS/delay.       */
    unsigned c;                 /* Cust. from class c removed.  */
    Population **P;             /* For marginal probabilities.  */
private:
    Vector<double> X;           /* Throughput per class.        */
    Vector<double> Z;           /* Think time per class.        */
    Vector<Interlock> IL;       /* Interlock information.       */
    unsigned long coreCount;    /* Number of iterations of step */
\end{verbatim}


Access to all instance variables for the MVA class is restricted to
members of the class.  
\begin{description}
\item[c] \texonly{---} Class of customer removed by
  \link{Linearizer}{sec:mva-linearizer}.  Not used by exact MVA and
  Schweitzer solvers, but defined here to allow for common expressions
  in the MVA \link{\code{step()}}{sec:mva-step} function.
\item[coreCount] \texonly{---} A counter that is incremented each time
  the \link{\code{step()}}{sec:mva-step} function is executed.
\item[K] \texonly{---} The maximum number of \bold{classes} at any station.
\item[L] \texonly{---} Queue \underline{L}ength.  Dimensioned by \bold{entry} and \bold{class}.
\item[M] \texonly{---} The number of \bold{stations} in the network.
\item[NCust] \texonly{---} A \link{vector}{sec:vector} containing the number of
  customers for each \bold{class}.  Classes range from 1 to \var{K}.
  \var{NCust[0]} is not used.
\item[P] \texonly{---} The \link{marginal
    probabilities}{sec:mva-marginal} for each station in the network.
  Population space is allocated only for those stations that need
  marginal probabilities (for example, multiservers).  \bold{Note:}
  \var{P[m][J][N]} is \var{PB[m][N]} where \var{J} is the number of
  servers at a multiserver.
\item[Q] \texonly{---} The \link{stations}{sec:server} in the network.  It is an
  array of pointers to stations and is indexed from 1 to \var{M}.
  \var{Q[0]} is not used.  
\item[U] \texonly{---} Station \underline{U}tilization.  Dimensioned by \bold{entry}
  and \bold{class}. 
\item[X] \texonly{---} the throughput for each \bold{class}.  It is
  valid for the last \link{population}{sec:population} vector solved
  (which is usually the maximum customer population.)
\item[Z] \texonly{---} The think time for each \bold{class}.
each time the MVA core step is executed.
\end{description}

\subsubsection{Variable Conventions}
The following variables are typically used throughout the MVA classes
either as arguments to, or locally within, a member function.

\label{sec:mva-conventions}
\begin{description}
\item[E] \texonly{---} The total number of entries for a station.
\item[e] \texonly{---} An entry index: \math[1 <= e <= E]{1 \leq e \leq E}.
\item[J] \texonly{---} The total number of servers at a multiserver
  station.
\item[j]  \texonly{---} A server index for multiservers: : \math[1 <=
  j <= J]{1 \leq j \leq J}.
\item[k] \texonly{---} A class (chain) index: \math[1 <= k <= K]{1
    \leq K \leq K}.
\item[m] \texonly{---} A station index: \math[1 <= m <= M]{1 \leq m
    \leq M}.
\item[N] \texonly{---} A Population \link{vector}{sec:vector}.
  Objects of type \link{\file{Population}}{sec:population} are indexed
  by this type.
\item[n] \texonly{---} A \emph{precomputed} index into the array that
  stores values for a given population \var{N}.
\end{description}


\subsubsection{Constructors for class MVA}

All solvers are created with the following type of call.  Subclasses
are expected to accept the same argument list.  The argument \var{m}
specifies the number of stations in the network.  \var{k} specifies
the number of classes.  \var{q} an array of pointers to objects of
type Server.  These objects are the stations in the queueing network.
Last, the vector \var{N} defines the maximum number of customers in
each class \var{k}.  Index 0 (zero) is NOT used in either \var{q} or
\var{N}.  Range checking does not take place for array \var{q}.

\begin{verbatim}
MVA( unsigned m, unsigned k, Server **q, Vector& N );
\end{verbatim}

\subsubsection{Instance Variable Access}

\begin{description}
\item[thinkTime] \texonly{---} Think Time.\\
  \code{double& thinkTime( const unsigned k );}

  The instance variable \var{Z}, (think time) can be set or retrieved
  with this function.
\end{description}

\subsubsection{Queries}

\begin{description}
\item[throughput] \texonly{---} Throughput.\\
  \code{double throughput( const unsigned e, const unsigned k ) const;}\\
  \code{double throughput( const unsigned e ) const;}

  Return the station's throughput for a given \bold{entry} and
  [optionally] \bold{class}.

\item[utilization] \texonly{---} Return the utilization of the station.\\
  \code{double utilization( const Server& station, const PopVector& N,
    const unsigned j ) const;} \\
  \code{double utilization( const Server& station, const unsigned k,
    const PopVector& N, const unsigned j ) const;} \\
  \code{double utilization( const unsigned m, const unsigned k, const
    PopVector& N ) const;} 

  The first of the three member functions returns the total
  utilization of \var{station} with one customer removed from class
  \var{j}.  The second member function returns the utilization for
  class \var{k} with one customer removed from class \var{j}.  The
  final version returns the utilization for class \var{k} at the full
  customer population \var{N}.

\item[sumOf_SL_m] \texonly{---} Queue length sum.\\
  \code{double sumOf_SL_m( const Server& s, const unsigned e, const
    PopVector &N, const unsigned j) const;}

  The function \code{sumOf_SL_m()} returns the sum of the service time
  multiplied by the queue length at station \var{s}, entry \var{e}, at
  the \link{population}{sec:population} \var{N} when one customer is
  removed from class \var{j}; i.e., \math[]{\sum_k s_{mk} L_m({\bf
      N}-e_{j})}.  This function is called from the
  \link{\file{Server}}{sec:server} class.  Its implementation depends on the
  MVA solver class.

\item[sumOf_p] \texonly{---} \math[Sum of Pi]{\sum (J_i-1-j)P_i(j,{\bf N} - e_j)}.\\
  \code{double sumOf_p( const Server& station, const PopVector &N, const
    unsigned j ) const}

  Return the \emph{something} with one customer removed from class
  \var{j} for population \var{N}. (See \link{Eqn. (2.19) and
    (3.8)}[~\Cite]{queue:reiser-80}, \link{Eqn.
    (2.1)}[~\Cite]{queue:conway-88}.)

\item[PB] \texonly{---} Pr(All Servers Busy)\\
  \code{double PB( const Server& station, const PopVector &N, const unsigned j ) const;}

  Return the probability that all servers are busy for a population of
  \var{N} and one customer removed from class \var{j}.

  Note: \math[PB(N) = P(J,N)]{{\it PB}_{m}({\bf N}) = P_{m}(J,{\bf N})}

\item[queueLength] \texonly{---} Find queue length.\\ \code{double
    queueLength( const Server&, const unsigned k, const PopVector& N,
    const unsigned j ) const;}

  Find the queue length at the server for class \var{k} with one customer
  from class \var{j} removed from the population \var{N}.  This function
  \emph{does not include} customers in service at the station. 

\item[iterations] \texonly{---} Number of core steps.\\
  \code{unsigned long iterations() const;}

  Returns the number of times the core MVA
  \link{\code{step}}{sec:mva-step} is called.

\item[offset] \texonly{---} Array offset calculation.\\
  \code{virtual unsigned offset( const PopVector& N ) const;} \\
  \code{virtual unsigned offset_e_j( const PopVector& N, const unsigned
    j ) const;} \\

  Calculate and return the \link{internal array
    offset}{sec:population-offset} for the
  \link{population}{sec:population} vector \var{N}.  These functions are
  defined by subclasses so that the proper offset function in the
  \link{\file{Population}}{sec:population} class can be called.  (Exact
  MVA uses \link{general populations}{sec:general-population} while the
  other solvers use \link{special populations}{sec:special-population}).

\end{description}

\subsubsection{Computation}
\begin{description}
\label{sec:mva-solve}
\item[solve] \texonly{---} Solve the model\\
\code{void solve();}

\code{Solve()} is used to solve the queueing network.  It invokes the
fuction \link{\code{step()}}{sec:mva-step} to perform the actual MVA
calculation.  Implemented by subclasses.


\label{sec:mva-step}
\item[step] \texonly{---} Core MVA step.\\
\code{void step( const PopVector& N );}

The \code{step()} member function is used to compute the queue lenght
and untilization at each station in the queueing network for the
population \var{N}.  It assumes that solution for the network exists
with one customer removed from each class \var{k}.  It first finds the
waiting time at each station (See Eqn.
\link{(1)}[~\Cite]{queue:chandy-82}). Next, the throughputs for each
class, \math[\var{X[k]}]{X_k}, are found.  This result is then used to
find the new queue length \math[\var{L}]{L_{mk}({\bf N})} (See Eqn.
\link{(6)}[~\Cite]{queue:chandy-82})

\begin{iftex}
\tex
\begin{equation}
  L_{mk}({\bf N}) = N_k \frac{ v_{mk} W_{mk}({\bf N}) }{
    \sum_{m^\prime} v_{m^\prime k} W_{m^\prime k}({\bf N})}
\end{equation}
\end{iftex}

The waiting time \math[\var{W}]{W_{mk}} is calculated by the class
\link{\code{Server}}[~(\Sec\Ref)]{sec:server}.  It supplies two
member functions, \link{\code{initStep()}}{sec:server-initstep} and
\code{wait()}.  \code{InitStep()} is called first and is used to
initialize any data prior to finding the new waiting time.
\code{Wait()} is used to find the waiting time at the current
population value.  Both functions are overridden by subclasses of
\file{Server} to provide the necessary functionality.
\begin{iftex}
  For example, Equation~\ref{eqn:delay} is used to find the waiting
  time for delay servers, Equation~\ref{eqn:fcfs} is used for First
  Come, First Served servers, and Equation~\ref{eqn:msfcfs} is used
  for multi-servers.
\tex
\begin{eqnarray}
W_{mk}({\bf N}) & = & s_{mk} \label{eqn:delay} \\
W_{mk}({\bf N}) & = & s_{mk} [ 1 + L_{mk}({\bf N} - e_k) ]
\label{eqn:fcfs} \\
W_{mk}({\bf N}) & = & \frac{ s_{mk} }{J_{m}} \left[ 1 + L_{mk}({\bf N} -
e_k) + \sum_{j=0}^{J_{m} - 2} (J_{m} - 1 - j) P(j,{\bf N} - e_k) \right]
\label{eqn:msfcfs} 
\end{eqnarray}
\end{iftex}

Note that \code{step()} may be called multiple times for a given
population (in particular, by the approximate solvers), while
\code{initStep()} is called only once.

\item[utilization] \texonly{---} Find utilization\\
\code{double utilization( const unsigned m, const unsigned e, const
  unsigned k, const double L ) const;}

Compute the utilization \math[\var{U}]{U_{mk}} for the various
stations and customer populations.  (See \link{Eqn.
  (7)}[~\Cite]{queue:chandy-82})

\begin{iftex}
\tex
\begin{equation}
  U_{mk}({\bf N}) = \frac{s_{mk} L_{mk}({\bf N})}{W_{m^\prime k}({\bf N})}
\end{equation}
\end{iftex}

\item[marginalProbabilities] \texonly{---} Compute P.\\
\code{void marginalProbabilities( const unsigned m, const PopVector& N );}

Compute the marginal probability that there are \var{j} customers at
station \var{m} for population \var{N}. The probabilities are stored
in \link{\var{P[m][j][N]}}{sec:mva-ivars}.  \bold{Note:} The
probability that all servers are busy is denoted as \var{PB} in the
literature, but is stored in \var{P} in the expected place (i.e.
\math[\var{j} = \var{M}]{j = J}).  (See Equations \link{(2.10) and
  (2.13) and (3.9)}[~\Cite]{queue:reiser-80} and \link{(2.4), (2.5)
  and (2.6)}[~\Cite]{queue:conway-88}).

\begin{iftex}
\tex
\begin{eqnarray}
P_{m}(j,{\bf N}) & = & \sum_{k=1}^{K} s_{mk}\lambda_{mk}({\bf N}))
P_{m}(j-1,{\bf N} - e_k) / j, \forall j, 0 < j < J \\
P_{m}(J,{\bf N}) & = & \sum_{k=1}^{K} s_{mk}\lambda_{mk}({\bf N}))
\left[ P_{m}(J,{\bf N} - e_k) + P_{m}(J-1,{\bf N} - e_k) \right] / J \\
P_{m}(0,{\bf N}) & = & 1 - \sum_{j=1}^{J} P_{m}(j,\bf{N})
\end{eqnarray}

Note that $P_{m}(J,{\bf N}) = PB_m({\bf N})$.
\end{iftex}

\end{description}

\subsubsection{Printing}
\begin{description}
\item[print] \texonly{---} Print all results.\\ \code{ostream& print(
    ostream& os );}

  Print the throughput and utilization for all
  \link{stations}{sec:server} using the
  \link{\code{printResults}}{sec:server-print} function.

\item[printL] \texonly{---} Print Queue Lengths.\\ \code{ostream&
    printL( ostream& os, const PopVector& N );}

  Print the value of queue length \link{\var{L}}{sec:server-ivars} at
  the \link{population}{sec:population} \var{N}.  This function is
  used to debug the solvers.

\item[printP] \texonly{---} Print marginal probabilities.\\ 
  \code{ostream& printZ( ostream& os );}

  Print marginal probabilities for all classes.

\item[printW] \texonly{---} Print waiting times.\\ \code{ostream&
    printL( ostream& os );}

  Print the value of waiting time \link{\var{W}}{sec:server-ivars}.
  This function is used to debug the solvers.

\item[printX] \texonly{---} Print throughputs.\\ 
  \code{ostream& printZ( ostream& os );}

  Print throughputs for all classes.

\item[printZ] \texonly{---} Print think times.\\ 
  \code{ostream& printZ( ostream& os );}

  Print think times for all classes.

\end{description}


\subsection{Class ExactMVA}
\label{sec:mva-exact}

The Exact MVA algorithm is taken from \link{Chandy and
  Neuse}[~\Cite]{queue:chandy-82} (Also refer to \link{Reiser and
  Lavenberg}[~\Cite]{queue:reiser-80} and \link{Lazowska et.
  al.}[~\Cite]{perf:lazowska:84}).  It sequences through all customer
populations, starting at \math[0,0,...0]{0_1,0_2,...,0_K}, calling the
\link{\code{step()}}{sec:mva-step} function for each.

\subsubsection{Queries}
\begin{description}
\item[offset] \texonly{---} Return array offset for population\\ 
  \code{unsigned offset( const PopVector& N ) const}\\ \code{unsigned
    offset_e_j( const PopVector& N, const unsigned j ) const}

  Return \link{population}{sec:population} array offset.

\end{description}
\subsubsection{Computation}
\begin{description}
\item[solve] \texonly{---} Solve model\\ \code{void solve()};

  Recursively solve for population vector \var{N} starting at a
  population of \math[0,0,...,0]{[0_1,0_2,...0_K]} to \var{nCust}.
  Solutions for each \link{population}{sec:population} vector in the
  population space are generated by calling \code{nextPop()}.  The
  dimensionality of \var{N} is limited by stack size.

Note: while is is possible to save space by not retaining
\link{population}{sec:population} values for the ``outermost'' class
(see \link{Lazowska}[~\Cite]{perf:lazowska:84}, Chapter 19), no
attempt is done so here.

\end{description}

\subsection{Class Schweitzer}
\label{sec:mva-schweitzer}

\subsubsection{Instance Variables}
\label{sec:mva-schweitzer-ivars}
\begin{verbatim}
private:
    double ***last_L;               /* For local comparison.        */
\end{verbatim}

\begin{description}
\item[last_L] \texonly{---} Previous value of the queue length \var{L}
  and is used to test for the termination condition.
\end{description}

\subsubsection{Initialization}
\begin{description}
\label{sec:mva-schweitzer-init-L}
\item[init_L] \texonly{---} Initialize Queue Lengths\\
\code{virtual void init_L( const PopVector & N);}

Find the queue length based on fraction \var{F} of class \var{k} jobs
at station for population \var{N} (See \link{Eqn.
  (11)}[~\Cite]{queue:chandy-82}).

\begin{iftex}
\tex
\begin{equation}
L_{mk}({\bf N} - e_j) = ({\bf N} - e_j)F_{mk}({\bf N}) + D_{mk}(\bf N)
\end{equation}
\end{iftex}

\item[init_U] \texonly{---} Initialize Utilizations\\
\code{virtual void init_U( const PopVector & );}

Initialize the utilization array \math[\var{U}]{U \forall m,e,k} for
population \math[\var{N} with one customer removed from each class in
turn]{N - e_j, 1 \leq j \leq K}.  \code{init_U()} is separated from
\link{\code{init_L()}}{sec:mva-schweitzer-init-L} to permit subclasses to override.

\end{description}
        
\subsubsection{Queries}
\begin{description}
\item[D_mekj] \texonly{---} Return D\\
\code{virtual double D_mekj(unsigned m,unsigned e,unsigned k,unsigned j);}

Return the change in the fraction of class \var{k} jobs at queue
\var{m} after one class \var{j} job has been removed.  For the
Schweitzer solver, the best estimate of \var{D} is zero.

\end{description}
\subsubsection{Computation}

\begin{description}
\item[solve] \texonly{---} Solve model using Schweitzer method.\\
\code{virtual void solve();}

\code{solve()} merely is a public interface to \code{core()}.

\label{sec:mva-schweitzer-core}
\item[core] \texonly{---} \\
\code{void core( const PopVector & );}

\code{Core()} is the basic iterative solver for Bard Schweitzer and
Linearizer algorithms.  For each iteraction, it first calls
\link{\code{init_L()}}{sec:mva-schweitzer-init-L} and \code{init_U()}
followed by \link{\code{step()}}{sec:mva-step}.

\end{description}

\subsection{Class Linearizer}
\label{sec:mva-linearizer}

The \link{linearizer
  algorithm}[~\Cite]{queue:chandy-82,queue:krzesinski-84,queue:conway-88}
improves upon the \link{Schweitzer}{sec:mva-schweitzer} method by
providing better estimates for the variable \math[\var{D}]{D_{mejk}}.
Linearizer calls the Schweitzer solver \math[K+1]{K+1} times per outer
loop: once each with one customer from class \math[\var{c}]{c}
removed, and once at the maximum customer population.  The outer loop
is executed three times.

\subsubsection{Instance Variables}
\label{sec:mva-linearizer-ivars}
\begin{verbatim}
protected:
    Population ***F;            /* Fraction of jobs at queue.   */
    Population ***saved_L;      /* Saved queue length.          */
    Population ***saved_U;      /* Saved utilization.           */
    double ****D;               /* Delta Fraction of jobs.      */
\end{verbatim}

\begin{description}
\item[D] \texonly{---} The change in the fraction of queue length for
  station \var{m}, entry \var{e}, class \var{k} when a class \var{j}
  customer is removed.
\item[F] \texonly{---} The fraction of class \var{k} jobs at station
  \var{m}, entry \var{e}. 
\item[saved_L] \texonly{---} Saved queue lengths.
\item[saved_U] \texonly{---} Saved station utilizations.
\end{description}

\subsubsection{Initialization}
\begin{description}
\label{sec:mva-linearizer-init-D}
\item[init_D] \texonly{---} Intialize \var{D}\\
\code{virtual void init_D( const PopVector & N );}

Initialize \var{D}, the difference in queue lengths when a customer
from class \var{j} is removed (See Eqn.
\link{(10)}[~\Cite]{queue:chandy-82}).  This member function is
overridden for the \link{fast linearizer}{sec:mva-fast} solver.
\begin{iftex}
\tex
\begin{equation}
D_{mkj}({\bf N}) = F_{mk}({\bf N} - e_j) - F_{mk}({\bf N})
\end{equation}
\end{iftex}

\end{description}

\subsubsection{Queries}

\begin{description}

\item[D_mekj] \texonly{---} Return D\\
code{double D_mekj( unsigned m, unsigned e, unsigned k, unsigned j );}

Return the change in the fraction of class \var{k} jobs at queue
\var{m} after one class \var{j} job has been removed.  This function
overrides the member function supplied by the \file{Schweitzer} class because
\var{D} is now computed.

\end{description}
        
\subsubsection{Computation}

\begin{description}
\item[solve] \texonly{---} Solve model using Linearizer method.\\
\code{virtual void solve();}

Solve the model using the linearizer solution strategy.  This member
function calls the Schweitzer solver,
\link{\code{core()}}{sec:mva-schweitzer-core} with one customer
removed from each class in turn and at the full population value.  The
instance variable \link{\var{c}}{sec:mva-ivars}, defined in the
abstract superclass, is set according to the class in which the
customer is removed.

The member function \link{\code{init_D()}}{sec:mva-linearizer-init-D}
is called to initialize the \var{D} array before each call to the
Schweitzer \code{core()} solver.  

\end{description}

\subsubsection{Input and Output}
\begin{description}
\item[print] \texonly{---} Printing fuctions\\
\code{ostream& printF( ostream&, const PopVector& N );}\\
\code{ostream& printD( ostream&, const PopVector& N );}

Print out the value of the instance variables \var{F} and \var{D} for
the population \var{N} respecitively.  Used for debugging only.

\end{description}

\subsection{Class Linearizer2}
\label{sec:mva-fast}

Algorithm is from \link{de Souza e Silva and Muntz}[~\Cite]{queue:deSouzaeSilva-90}.

\subsubsection{Instance Variables}
\label{sec:mva-fast-ivars}
\begin{verbatim}
private:
    SpecialPop **Lm;            /* Queue length sum (Fast lin.) */
    SpecialPop **Um;            /* Queue length sum (Fast lin.) */
    double ***D_k;              /* Sum over k.                  */
\end{verbatim}

\begin{description}
\item[D_k] \texonly{---} \math[Sum over class \var{k} of
  \var{D}]{D^\prime_{mj} = \sum_{k=1}^{K} N_k D_{mkj}({\bf N})}.
\item[Lm] \texonly{---} Queue \underline{L}ength summed over all stations.  Used by
  the \link{fast linearizer}{sec:mva-fast}.  Dimensioned by \bold{entry}.
\item[Um] \texonly{---} Station \underline{U}tilization.  Used by the \link{fast
    linearizer}{sec:mva-fast}.  Dimensioned by \bold{entry}. 
\end{description}
\subsubsection{Initialization}

\begin{description}
\label{sec:mva-fast-init-D}
\item[init_D] \texonly{---} Initialize D array.\\
\code{void init_D( const PopVector & );}

Initialize \var{D}, the difference in queue lengths when a customer
from class \var{j} is removed.  This version of the \code{init_D()}
member function also initializes \math[\var{D_k}]{D^\prime_{mj}}.


\label{sec:mva-linearizer-init-L}
\item[init_L] \texonly{---} Initialize Queue lengths.\\
\code{void init_L( const PopVector & );}

Pre-compute \math[\var{S} times \var{Lm}]{S \cdot L_{mk}} and save.
\math[\var{S} times \var{L}]{S \cdot L_{mk}} must be pre-computed
because it relies on the old values of \math[\var{Lm}]{L_{mk}} for all
populations of \math[\var{N}]{\bf N}.  The expressions from \link{de
  Souza e Silva and Muntz}[~\Cite]{queue:deSouzaeSilva-90} (Equations
(8) and (9)) do not multiply through by service times --- this version
does to simplify the expressions for the \link{High Variability First
  Come, First Served}{sec:server-hvfcfs} Server.

\begin{iftex}
\tex
\begin{eqnarray}
L_{m}({\bf N} - e_j ) & = & L_{m}({\bf N}) - \frac{L_{m}({\bf N})}{N_j} +
D^\prime_{mj}({\bf N}) - D_{mjj}({\bf N}) \\
L_{m}({\bf M} - e_j ) & = & L_{m}({\bf M}) - \frac{L_{m}({\bf M})}{M_j} +
D^\prime_{mj}({\bf M}) - D_{mjj}({\bf M}) + - D_{mcj}({\bf M}) \\
{\bf M} & = & {\bf N} - e_c \nonumber
\end{eqnarray}
\end{iftex}

\item[init_U] \texonly{---} Initialize Utilizations.\\
\code{void init_U( const PopVector & );}

Initialize the utilization array \math[\var{U}]{U \forall m,e,k} for
population \math[\var{N}]{N - e_j, 1 \leq j \leq K}.

\end{description}
\subsubsection{Queries}

\begin{description}

\item[sumOf_SL_m] \texonly{---} Queue length sum.\\
\code{double sumOf_SL_m( const Server& s, const unsigned e, const
  PopVector &N, const unsigned j ) const;}

Return the pre-computed value of \var{Lm} for entry \var{e}, class
\var{j}.  This value was calculated in
\link{\code{init_L()}}{sec:mva-linearizer-init-L} earlier.

\end{description}

\C Local Variables: 
\C mode: latex
\C TeX-master: "lqns"
\C End: 

    \caption{MVA class hierarchy}
    \H \htmlimage{mva.gif}
  \end{center}
\end{figure}

This class relies on \link{\file{Server}}[ (\Sec\Ref)]{sec:server} for the waiting
time calculations for each station.

\htmlrule
\subsection{Important Performance Notes}
\label{sec:mva-cautions}

Common expression hoisting has been performed to reduce computational
costs.  The expressions are marked with the comment ``\code{Hoist}''
in the source code.  It is assumed that the size and the maximum
number of customers for all \link{\file{Population}}{sec:population}
vectors for all stations is the same so that the offset for a given
population value \var{N} is the same for all objects of type
\file{Population}.  The offset computation is hoisted out of the loops
where possible.  The integer variable \var{n} represents the array
offset for the customer population \var{N}, i.e., \code{n =
  population.offset(N)}.  The integer variable \var{Nej} represents
the array offset of customer population \var{N} with one customer of
class \emph{j} removed.  

The functions in class
\link{\file{Population}}[~(\Sec\Ref)]{sec:population} exported to aid
hoisting are:
\begin{itemize}
\item \code{unsigned offset( const PopVector & N ) const;}
\item \code{unsigned offset_e_j( const PopVector & N, const unsigned j );}
\item \code{unsigned offset_e_c_e_j( const unsigned c, const unsigned j);}
\end{itemize}

The instance variable \link{\var{c}}{sec:mva-ivars} used in the
\link{Linearizer solver}[ (\Sec\Ref)]{sec:mva-linearizer} and subclasses
is also very special.  It represents the class of the customer being
removed by the linearizer algorithm.  This variable is not be
explicitly visible.  However, it is used, so \bold{DO NOT USE} \var{c}
as a local variable \bold{ANYWHERE}.  It is used in \bold{ALL CLASSES}
including MVA and especially \code{MVA::step()}.

\subsection{Class MVA --- Abstract Superclass}

\subsubsection{Instance Variables}
\label{sec:mva-ivars}

\begin{verbatim}
public:
    static int boundsLimit;     /* Enable bounds limiting.      */
    Population ***L;            /* Queue length.                */
    Population ***U;            /* Station utilization.         */
protected:
    const PopVector NCust;      /* Number of Customers.         */
    const unsigned M;           /* Number of stations.          */
    const unsigned K;           /* Number of classes.           */
    Server **Q;                 /* Queue type.  SS/delay.       */
    unsigned c;                 /* Cust. from class c removed.  */
    Population **P;             /* For marginal probabilities.  */
private:
    Vector<double> X;           /* Throughput per class.        */
    Vector<double> Z;           /* Think time per class.        */
    Vector<Interlock> IL;       /* Interlock information.       */
    unsigned long coreCount;    /* Number of iterations of step */
\end{verbatim}


Access to all instance variables for the MVA class is restricted to
members of the class.  
\begin{description}
\item[c] \texonly{---} Class of customer removed by
  \link{Linearizer}{sec:mva-linearizer}.  Not used by exact MVA and
  Schweitzer solvers, but defined here to allow for common expressions
  in the MVA \link{\code{step()}}{sec:mva-step} function.
\item[coreCount] \texonly{---} A counter that is incremented each time
  the \link{\code{step()}}{sec:mva-step} function is executed.
\item[K] \texonly{---} The maximum number of \bold{classes} at any station.
\item[L] \texonly{---} Queue \underline{L}ength.  Dimensioned by \bold{entry} and \bold{class}.
\item[M] \texonly{---} The number of \bold{stations} in the network.
\item[NCust] \texonly{---} A \link{vector}{sec:vector} containing the number of
  customers for each \bold{class}.  Classes range from 1 to \var{K}.
  \var{NCust[0]} is not used.
\item[P] \texonly{---} The \link{marginal
    probabilities}{sec:mva-marginal} for each station in the network.
  Population space is allocated only for those stations that need
  marginal probabilities (for example, multiservers).  \bold{Note:}
  \var{P[m][J][N]} is \var{PB[m][N]} where \var{J} is the number of
  servers at a multiserver.
\item[Q] \texonly{---} The \link{stations}{sec:server} in the network.  It is an
  array of pointers to stations and is indexed from 1 to \var{M}.
  \var{Q[0]} is not used.  
\item[U] \texonly{---} Station \underline{U}tilization.  Dimensioned by \bold{entry}
  and \bold{class}. 
\item[X] \texonly{---} the throughput for each \bold{class}.  It is
  valid for the last \link{population}{sec:population} vector solved
  (which is usually the maximum customer population.)
\item[Z] \texonly{---} The think time for each \bold{class}.
each time the MVA core step is executed.
\end{description}

\subsubsection{Variable Conventions}
The following variables are typically used throughout the MVA classes
either as arguments to, or locally within, a member function.

\label{sec:mva-conventions}
\begin{description}
\item[E] \texonly{---} The total number of entries for a station.
\item[e] \texonly{---} An entry index: \math[1 <= e <= E]{1 \leq e \leq E}.
\item[J] \texonly{---} The total number of servers at a multiserver
  station.
\item[j]  \texonly{---} A server index for multiservers: : \math[1 <=
  j <= J]{1 \leq j \leq J}.
\item[k] \texonly{---} A class (chain) index: \math[1 <= k <= K]{1
    \leq K \leq K}.
\item[m] \texonly{---} A station index: \math[1 <= m <= M]{1 \leq m
    \leq M}.
\item[N] \texonly{---} A Population \link{vector}{sec:vector}.
  Objects of type \link{\file{Population}}{sec:population} are indexed
  by this type.
\item[n] \texonly{---} A \emph{precomputed} index into the array that
  stores values for a given population \var{N}.
\end{description}


\subsubsection{Constructors for class MVA}

All solvers are created with the following type of call.  Subclasses
are expected to accept the same argument list.  The argument \var{m}
specifies the number of stations in the network.  \var{k} specifies
the number of classes.  \var{q} an array of pointers to objects of
type Server.  These objects are the stations in the queueing network.
Last, the vector \var{N} defines the maximum number of customers in
each class \var{k}.  Index 0 (zero) is NOT used in either \var{q} or
\var{N}.  Range checking does not take place for array \var{q}.

\begin{verbatim}
MVA( unsigned m, unsigned k, Server **q, Vector& N );
\end{verbatim}

\subsubsection{Instance Variable Access}

\begin{description}
\item[thinkTime] \texonly{---} Think Time.\\
  \code{double& thinkTime( const unsigned k );}

  The instance variable \var{Z}, (think time) can be set or retrieved
  with this function.
\end{description}

\subsubsection{Queries}

\begin{description}
\item[throughput] \texonly{---} Throughput.\\
  \code{double throughput( const unsigned e, const unsigned k ) const;}\\
  \code{double throughput( const unsigned e ) const;}

  Return the station's throughput for a given \bold{entry} and
  [optionally] \bold{class}.

\item[utilization] \texonly{---} Return the utilization of the station.\\
  \code{double utilization( const Server& station, const PopVector& N,
    const unsigned j ) const;} \\
  \code{double utilization( const Server& station, const unsigned k,
    const PopVector& N, const unsigned j ) const;} \\
  \code{double utilization( const unsigned m, const unsigned k, const
    PopVector& N ) const;} 

  The first of the three member functions returns the total
  utilization of \var{station} with one customer removed from class
  \var{j}.  The second member function returns the utilization for
  class \var{k} with one customer removed from class \var{j}.  The
  final version returns the utilization for class \var{k} at the full
  customer population \var{N}.

\item[sumOf_SL_m] \texonly{---} Queue length sum.\\
  \code{double sumOf_SL_m( const Server& s, const unsigned e, const
    PopVector &N, const unsigned j) const;}

  The function \code{sumOf_SL_m()} returns the sum of the service time
  multiplied by the queue length at station \var{s}, entry \var{e}, at
  the \link{population}{sec:population} \var{N} when one customer is
  removed from class \var{j}; i.e., \math[]{\sum_k s_{mk} L_m({\bf
      N}-e_{j})}.  This function is called from the
  \link{\file{Server}}{sec:server} class.  Its implementation depends on the
  MVA solver class.

\item[sumOf_p] \texonly{---} \math[Sum of Pi]{\sum (J_i-1-j)P_i(j,{\bf N} - e_j)}.\\
  \code{double sumOf_p( const Server& station, const PopVector &N, const
    unsigned j ) const}

  Return the \emph{something} with one customer removed from class
  \var{j} for population \var{N}. (See \link{Eqn. (2.19) and
    (3.8)}[~\Cite]{queue:reiser-80}, \link{Eqn.
    (2.1)}[~\Cite]{queue:conway-88}.)

\item[PB] \texonly{---} Pr(All Servers Busy)\\
  \code{double PB( const Server& station, const PopVector &N, const unsigned j ) const;}

  Return the probability that all servers are busy for a population of
  \var{N} and one customer removed from class \var{j}.

  Note: \math[PB(N) = P(J,N)]{{\it PB}_{m}({\bf N}) = P_{m}(J,{\bf N})}

\item[queueLength] \texonly{---} Find queue length.\\ \code{double
    queueLength( const Server&, const unsigned k, const PopVector& N,
    const unsigned j ) const;}

  Find the queue length at the server for class \var{k} with one customer
  from class \var{j} removed from the population \var{N}.  This function
  \emph{does not include} customers in service at the station. 

\item[iterations] \texonly{---} Number of core steps.\\
  \code{unsigned long iterations() const;}

  Returns the number of times the core MVA
  \link{\code{step}}{sec:mva-step} is called.

\item[offset] \texonly{---} Array offset calculation.\\
  \code{virtual unsigned offset( const PopVector& N ) const;} \\
  \code{virtual unsigned offset_e_j( const PopVector& N, const unsigned
    j ) const;} \\

  Calculate and return the \link{internal array
    offset}{sec:population-offset} for the
  \link{population}{sec:population} vector \var{N}.  These functions are
  defined by subclasses so that the proper offset function in the
  \link{\file{Population}}{sec:population} class can be called.  (Exact
  MVA uses \link{general populations}{sec:general-population} while the
  other solvers use \link{special populations}{sec:special-population}).

\end{description}

\subsubsection{Computation}
\begin{description}
\label{sec:mva-solve}
\item[solve] \texonly{---} Solve the model\\
\code{void solve();}

\code{Solve()} is used to solve the queueing network.  It invokes the
fuction \link{\code{step()}}{sec:mva-step} to perform the actual MVA
calculation.  Implemented by subclasses.


\label{sec:mva-step}
\item[step] \texonly{---} Core MVA step.\\
\code{void step( const PopVector& N );}

The \code{step()} member function is used to compute the queue lenght
and untilization at each station in the queueing network for the
population \var{N}.  It assumes that solution for the network exists
with one customer removed from each class \var{k}.  It first finds the
waiting time at each station (See Eqn.
\link{(1)}[~\Cite]{queue:chandy-82}). Next, the throughputs for each
class, \math[\var{X[k]}]{X_k}, are found.  This result is then used to
find the new queue length \math[\var{L}]{L_{mk}({\bf N})} (See Eqn.
\link{(6)}[~\Cite]{queue:chandy-82})

\begin{iftex}
\tex
\begin{equation}
  L_{mk}({\bf N}) = N_k \frac{ v_{mk} W_{mk}({\bf N}) }{
    \sum_{m^\prime} v_{m^\prime k} W_{m^\prime k}({\bf N})}
\end{equation}
\end{iftex}

The waiting time \math[\var{W}]{W_{mk}} is calculated by the class
\link{\code{Server}}[~(\Sec\Ref)]{sec:server}.  It supplies two
member functions, \link{\code{initStep()}}{sec:server-initstep} and
\code{wait()}.  \code{InitStep()} is called first and is used to
initialize any data prior to finding the new waiting time.
\code{Wait()} is used to find the waiting time at the current
population value.  Both functions are overridden by subclasses of
\file{Server} to provide the necessary functionality.
\begin{iftex}
  For example, Equation~\ref{eqn:delay} is used to find the waiting
  time for delay servers, Equation~\ref{eqn:fcfs} is used for First
  Come, First Served servers, and Equation~\ref{eqn:msfcfs} is used
  for multi-servers.
\tex
\begin{eqnarray}
W_{mk}({\bf N}) & = & s_{mk} \label{eqn:delay} \\
W_{mk}({\bf N}) & = & s_{mk} [ 1 + L_{mk}({\bf N} - e_k) ]
\label{eqn:fcfs} \\
W_{mk}({\bf N}) & = & \frac{ s_{mk} }{J_{m}} \left[ 1 + L_{mk}({\bf N} -
e_k) + \sum_{j=0}^{J_{m} - 2} (J_{m} - 1 - j) P(j,{\bf N} - e_k) \right]
\label{eqn:msfcfs} 
\end{eqnarray}
\end{iftex}

Note that \code{step()} may be called multiple times for a given
population (in particular, by the approximate solvers), while
\code{initStep()} is called only once.

\item[utilization] \texonly{---} Find utilization\\
\code{double utilization( const unsigned m, const unsigned e, const
  unsigned k, const double L ) const;}

Compute the utilization \math[\var{U}]{U_{mk}} for the various
stations and customer populations.  (See \link{Eqn.
  (7)}[~\Cite]{queue:chandy-82})

\begin{iftex}
\tex
\begin{equation}
  U_{mk}({\bf N}) = \frac{s_{mk} L_{mk}({\bf N})}{W_{m^\prime k}({\bf N})}
\end{equation}
\end{iftex}

\item[marginalProbabilities] \texonly{---} Compute P.\\
\code{void marginalProbabilities( const unsigned m, const PopVector& N );}

Compute the marginal probability that there are \var{j} customers at
station \var{m} for population \var{N}. The probabilities are stored
in \link{\var{P[m][j][N]}}{sec:mva-ivars}.  \bold{Note:} The
probability that all servers are busy is denoted as \var{PB} in the
literature, but is stored in \var{P} in the expected place (i.e.
\math[\var{j} = \var{M}]{j = J}).  (See Equations \link{(2.10) and
  (2.13) and (3.9)}[~\Cite]{queue:reiser-80} and \link{(2.4), (2.5)
  and (2.6)}[~\Cite]{queue:conway-88}).

\begin{iftex}
\tex
\begin{eqnarray}
P_{m}(j,{\bf N}) & = & \sum_{k=1}^{K} s_{mk}\lambda_{mk}({\bf N}))
P_{m}(j-1,{\bf N} - e_k) / j, \forall j, 0 < j < J \\
P_{m}(J,{\bf N}) & = & \sum_{k=1}^{K} s_{mk}\lambda_{mk}({\bf N}))
\left[ P_{m}(J,{\bf N} - e_k) + P_{m}(J-1,{\bf N} - e_k) \right] / J \\
P_{m}(0,{\bf N}) & = & 1 - \sum_{j=1}^{J} P_{m}(j,\bf{N})
\end{eqnarray}

Note that $P_{m}(J,{\bf N}) = PB_m({\bf N})$.
\end{iftex}

\end{description}

\subsubsection{Printing}
\begin{description}
\item[print] \texonly{---} Print all results.\\ \code{ostream& print(
    ostream& os );}

  Print the throughput and utilization for all
  \link{stations}{sec:server} using the
  \link{\code{printResults}}{sec:server-print} function.

\item[printL] \texonly{---} Print Queue Lengths.\\ \code{ostream&
    printL( ostream& os, const PopVector& N );}

  Print the value of queue length \link{\var{L}}{sec:server-ivars} at
  the \link{population}{sec:population} \var{N}.  This function is
  used to debug the solvers.

\item[printP] \texonly{---} Print marginal probabilities.\\ 
  \code{ostream& printZ( ostream& os );}

  Print marginal probabilities for all classes.

\item[printW] \texonly{---} Print waiting times.\\ \code{ostream&
    printL( ostream& os );}

  Print the value of waiting time \link{\var{W}}{sec:server-ivars}.
  This function is used to debug the solvers.

\item[printX] \texonly{---} Print throughputs.\\ 
  \code{ostream& printZ( ostream& os );}

  Print throughputs for all classes.

\item[printZ] \texonly{---} Print think times.\\ 
  \code{ostream& printZ( ostream& os );}

  Print think times for all classes.

\end{description}


\subsection{Class ExactMVA}
\label{sec:mva-exact}

The Exact MVA algorithm is taken from \link{Chandy and
  Neuse}[~\Cite]{queue:chandy-82} (Also refer to \link{Reiser and
  Lavenberg}[~\Cite]{queue:reiser-80} and \link{Lazowska et.
  al.}[~\Cite]{perf:lazowska:84}).  It sequences through all customer
populations, starting at \math[0,0,...0]{0_1,0_2,...,0_K}, calling the
\link{\code{step()}}{sec:mva-step} function for each.

\subsubsection{Queries}
\begin{description}
\item[offset] \texonly{---} Return array offset for population\\ 
  \code{unsigned offset( const PopVector& N ) const}\\ \code{unsigned
    offset_e_j( const PopVector& N, const unsigned j ) const}

  Return \link{population}{sec:population} array offset.

\end{description}
\subsubsection{Computation}
\begin{description}
\item[solve] \texonly{---} Solve model\\ \code{void solve()};

  Recursively solve for population vector \var{N} starting at a
  population of \math[0,0,...,0]{[0_1,0_2,...0_K]} to \var{nCust}.
  Solutions for each \link{population}{sec:population} vector in the
  population space are generated by calling \code{nextPop()}.  The
  dimensionality of \var{N} is limited by stack size.

Note: while is is possible to save space by not retaining
\link{population}{sec:population} values for the ``outermost'' class
(see \link{Lazowska}[~\Cite]{perf:lazowska:84}, Chapter 19), no
attempt is done so here.

\end{description}

\subsection{Class Schweitzer}
\label{sec:mva-schweitzer}

\subsubsection{Instance Variables}
\label{sec:mva-schweitzer-ivars}
\begin{verbatim}
private:
    double ***last_L;               /* For local comparison.        */
\end{verbatim}

\begin{description}
\item[last_L] \texonly{---} Previous value of the queue length \var{L}
  and is used to test for the termination condition.
\end{description}

\subsubsection{Initialization}
\begin{description}
\label{sec:mva-schweitzer-init-L}
\item[init_L] \texonly{---} Initialize Queue Lengths\\
\code{virtual void init_L( const PopVector & N);}

Find the queue length based on fraction \var{F} of class \var{k} jobs
at station for population \var{N} (See \link{Eqn.
  (11)}[~\Cite]{queue:chandy-82}).

\begin{iftex}
\tex
\begin{equation}
L_{mk}({\bf N} - e_j) = ({\bf N} - e_j)F_{mk}({\bf N}) + D_{mk}(\bf N)
\end{equation}
\end{iftex}

\item[init_U] \texonly{---} Initialize Utilizations\\
\code{virtual void init_U( const PopVector & );}

Initialize the utilization array \math[\var{U}]{U \forall m,e,k} for
population \math[\var{N} with one customer removed from each class in
turn]{N - e_j, 1 \leq j \leq K}.  \code{init_U()} is separated from
\link{\code{init_L()}}{sec:mva-schweitzer-init-L} to permit subclasses to override.

\end{description}
        
\subsubsection{Queries}
\begin{description}
\item[D_mekj] \texonly{---} Return D\\
\code{virtual double D_mekj(unsigned m,unsigned e,unsigned k,unsigned j);}

Return the change in the fraction of class \var{k} jobs at queue
\var{m} after one class \var{j} job has been removed.  For the
Schweitzer solver, the best estimate of \var{D} is zero.

\end{description}
\subsubsection{Computation}

\begin{description}
\item[solve] \texonly{---} Solve model using Schweitzer method.\\
\code{virtual void solve();}

\code{solve()} merely is a public interface to \code{core()}.

\label{sec:mva-schweitzer-core}
\item[core] \texonly{---} \\
\code{void core( const PopVector & );}

\code{Core()} is the basic iterative solver for Bard Schweitzer and
Linearizer algorithms.  For each iteraction, it first calls
\link{\code{init_L()}}{sec:mva-schweitzer-init-L} and \code{init_U()}
followed by \link{\code{step()}}{sec:mva-step}.

\end{description}

\subsection{Class Linearizer}
\label{sec:mva-linearizer}

The \link{linearizer
  algorithm}[~\Cite]{queue:chandy-82,queue:krzesinski-84,queue:conway-88}
improves upon the \link{Schweitzer}{sec:mva-schweitzer} method by
providing better estimates for the variable \math[\var{D}]{D_{mejk}}.
Linearizer calls the Schweitzer solver \math[K+1]{K+1} times per outer
loop: once each with one customer from class \math[\var{c}]{c}
removed, and once at the maximum customer population.  The outer loop
is executed three times.

\subsubsection{Instance Variables}
\label{sec:mva-linearizer-ivars}
\begin{verbatim}
protected:
    Population ***F;            /* Fraction of jobs at queue.   */
    Population ***saved_L;      /* Saved queue length.          */
    Population ***saved_U;      /* Saved utilization.           */
    double ****D;               /* Delta Fraction of jobs.      */
\end{verbatim}

\begin{description}
\item[D] \texonly{---} The change in the fraction of queue length for
  station \var{m}, entry \var{e}, class \var{k} when a class \var{j}
  customer is removed.
\item[F] \texonly{---} The fraction of class \var{k} jobs at station
  \var{m}, entry \var{e}. 
\item[saved_L] \texonly{---} Saved queue lengths.
\item[saved_U] \texonly{---} Saved station utilizations.
\end{description}

\subsubsection{Initialization}
\begin{description}
\label{sec:mva-linearizer-init-D}
\item[init_D] \texonly{---} Intialize \var{D}\\
\code{virtual void init_D( const PopVector & N );}

Initialize \var{D}, the difference in queue lengths when a customer
from class \var{j} is removed (See Eqn.
\link{(10)}[~\Cite]{queue:chandy-82}).  This member function is
overridden for the \link{fast linearizer}{sec:mva-fast} solver.
\begin{iftex}
\tex
\begin{equation}
D_{mkj}({\bf N}) = F_{mk}({\bf N} - e_j) - F_{mk}({\bf N})
\end{equation}
\end{iftex}

\end{description}

\subsubsection{Queries}

\begin{description}

\item[D_mekj] \texonly{---} Return D\\
code{double D_mekj( unsigned m, unsigned e, unsigned k, unsigned j );}

Return the change in the fraction of class \var{k} jobs at queue
\var{m} after one class \var{j} job has been removed.  This function
overrides the member function supplied by the \file{Schweitzer} class because
\var{D} is now computed.

\end{description}
        
\subsubsection{Computation}

\begin{description}
\item[solve] \texonly{---} Solve model using Linearizer method.\\
\code{virtual void solve();}

Solve the model using the linearizer solution strategy.  This member
function calls the Schweitzer solver,
\link{\code{core()}}{sec:mva-schweitzer-core} with one customer
removed from each class in turn and at the full population value.  The
instance variable \link{\var{c}}{sec:mva-ivars}, defined in the
abstract superclass, is set according to the class in which the
customer is removed.

The member function \link{\code{init_D()}}{sec:mva-linearizer-init-D}
is called to initialize the \var{D} array before each call to the
Schweitzer \code{core()} solver.  

\end{description}

\subsubsection{Input and Output}
\begin{description}
\item[print] \texonly{---} Printing fuctions\\
\code{ostream& printF( ostream&, const PopVector& N );}\\
\code{ostream& printD( ostream&, const PopVector& N );}

Print out the value of the instance variables \var{F} and \var{D} for
the population \var{N} respecitively.  Used for debugging only.

\end{description}

\subsection{Class Linearizer2}
\label{sec:mva-fast}

Algorithm is from \link{de Souza e Silva and Muntz}[~\Cite]{queue:deSouzaeSilva-90}.

\subsubsection{Instance Variables}
\label{sec:mva-fast-ivars}
\begin{verbatim}
private:
    SpecialPop **Lm;            /* Queue length sum (Fast lin.) */
    SpecialPop **Um;            /* Queue length sum (Fast lin.) */
    double ***D_k;              /* Sum over k.                  */
\end{verbatim}

\begin{description}
\item[D_k] \texonly{---} \math[Sum over class \var{k} of
  \var{D}]{D^\prime_{mj} = \sum_{k=1}^{K} N_k D_{mkj}({\bf N})}.
\item[Lm] \texonly{---} Queue \underline{L}ength summed over all stations.  Used by
  the \link{fast linearizer}{sec:mva-fast}.  Dimensioned by \bold{entry}.
\item[Um] \texonly{---} Station \underline{U}tilization.  Used by the \link{fast
    linearizer}{sec:mva-fast}.  Dimensioned by \bold{entry}. 
\end{description}
\subsubsection{Initialization}

\begin{description}
\label{sec:mva-fast-init-D}
\item[init_D] \texonly{---} Initialize D array.\\
\code{void init_D( const PopVector & );}

Initialize \var{D}, the difference in queue lengths when a customer
from class \var{j} is removed.  This version of the \code{init_D()}
member function also initializes \math[\var{D_k}]{D^\prime_{mj}}.


\label{sec:mva-linearizer-init-L}
\item[init_L] \texonly{---} Initialize Queue lengths.\\
\code{void init_L( const PopVector & );}

Pre-compute \math[\var{S} times \var{Lm}]{S \cdot L_{mk}} and save.
\math[\var{S} times \var{L}]{S \cdot L_{mk}} must be pre-computed
because it relies on the old values of \math[\var{Lm}]{L_{mk}} for all
populations of \math[\var{N}]{\bf N}.  The expressions from \link{de
  Souza e Silva and Muntz}[~\Cite]{queue:deSouzaeSilva-90} (Equations
(8) and (9)) do not multiply through by service times --- this version
does to simplify the expressions for the \link{High Variability First
  Come, First Served}{sec:server-hvfcfs} Server.

\begin{iftex}
\tex
\begin{eqnarray}
L_{m}({\bf N} - e_j ) & = & L_{m}({\bf N}) - \frac{L_{m}({\bf N})}{N_j} +
D^\prime_{mj}({\bf N}) - D_{mjj}({\bf N}) \\
L_{m}({\bf M} - e_j ) & = & L_{m}({\bf M}) - \frac{L_{m}({\bf M})}{M_j} +
D^\prime_{mj}({\bf M}) - D_{mjj}({\bf M}) + - D_{mcj}({\bf M}) \\
{\bf M} & = & {\bf N} - e_c \nonumber
\end{eqnarray}
\end{iftex}

\item[init_U] \texonly{---} Initialize Utilizations.\\
\code{void init_U( const PopVector & );}

Initialize the utilization array \math[\var{U}]{U \forall m,e,k} for
population \math[\var{N}]{N - e_j, 1 \leq j \leq K}.

\end{description}
\subsubsection{Queries}

\begin{description}

\item[sumOf_SL_m] \texonly{---} Queue length sum.\\
\code{double sumOf_SL_m( const Server& s, const unsigned e, const
  PopVector &N, const unsigned j ) const;}

Return the pre-computed value of \var{Lm} for entry \var{e}, class
\var{j}.  This value was calculated in
\link{\code{init_L()}}{sec:mva-linearizer-init-L} earlier.

\end{description}

\C Local Variables: 
\C mode: latex
\C TeX-master: "lqns"
\C End: 

    \caption{MVA class hierarchy}
    \H \htmlimage{mva.gif}
  \end{center}
\end{figure}

This class relies on \link{\file{Server}}[ (\Sec\Ref)]{sec:server} for the waiting
time calculations for each station.

\htmlrule
\subsection{Important Performance Notes}
\label{sec:mva-cautions}

Common expression hoisting has been performed to reduce computational
costs.  The expressions are marked with the comment ``\code{Hoist}''
in the source code.  It is assumed that the size and the maximum
number of customers for all \link{\file{Population}}{sec:population}
vectors for all stations is the same so that the offset for a given
population value \var{N} is the same for all objects of type
\file{Population}.  The offset computation is hoisted out of the loops
where possible.  The integer variable \var{n} represents the array
offset for the customer population \var{N}, i.e., \code{n =
  population.offset(N)}.  The integer variable \var{Nej} represents
the array offset of customer population \var{N} with one customer of
class \emph{j} removed.  

The functions in class
\link{\file{Population}}[~(\Sec\Ref)]{sec:population} exported to aid
hoisting are:
\begin{itemize}
\item \code{unsigned offset( const PopVector & N ) const;}
\item \code{unsigned offset_e_j( const PopVector & N, const unsigned j );}
\item \code{unsigned offset_e_c_e_j( const unsigned c, const unsigned j);}
\end{itemize}

The instance variable \link{\var{c}}{sec:mva-ivars} used in the
\link{Linearizer solver}[ (\Sec\Ref)]{sec:mva-linearizer} and subclasses
is also very special.  It represents the class of the customer being
removed by the linearizer algorithm.  This variable is not be
explicitly visible.  However, it is used, so \bold{DO NOT USE} \var{c}
as a local variable \bold{ANYWHERE}.  It is used in \bold{ALL CLASSES}
including MVA and especially \code{MVA::step()}.

\subsection{Class MVA --- Abstract Superclass}

\subsubsection{Instance Variables}
\label{sec:mva-ivars}

\begin{verbatim}
public:
    static int boundsLimit;     /* Enable bounds limiting.      */
    Population ***L;            /* Queue length.                */
    Population ***U;            /* Station utilization.         */
protected:
    const PopVector NCust;      /* Number of Customers.         */
    const unsigned M;           /* Number of stations.          */
    const unsigned K;           /* Number of classes.           */
    Server **Q;                 /* Queue type.  SS/delay.       */
    unsigned c;                 /* Cust. from class c removed.  */
    Population **P;             /* For marginal probabilities.  */
private:
    Vector<double> X;           /* Throughput per class.        */
    Vector<double> Z;           /* Think time per class.        */
    Vector<Interlock> IL;       /* Interlock information.       */
    unsigned long coreCount;    /* Number of iterations of step */
\end{verbatim}


Access to all instance variables for the MVA class is restricted to
members of the class.  
\begin{description}
\item[c] \texonly{---} Class of customer removed by
  \link{Linearizer}{sec:mva-linearizer}.  Not used by exact MVA and
  Schweitzer solvers, but defined here to allow for common expressions
  in the MVA \link{\code{step()}}{sec:mva-step} function.
\item[coreCount] \texonly{---} A counter that is incremented each time
  the \link{\code{step()}}{sec:mva-step} function is executed.
\item[K] \texonly{---} The maximum number of \bold{classes} at any station.
\item[L] \texonly{---} Queue \underline{L}ength.  Dimensioned by \bold{entry} and \bold{class}.
\item[M] \texonly{---} The number of \bold{stations} in the network.
\item[NCust] \texonly{---} A \link{vector}{sec:vector} containing the number of
  customers for each \bold{class}.  Classes range from 1 to \var{K}.
  \var{NCust[0]} is not used.
\item[P] \texonly{---} The \link{marginal
    probabilities}{sec:mva-marginal} for each station in the network.
  Population space is allocated only for those stations that need
  marginal probabilities (for example, multiservers).  \bold{Note:}
  \var{P[m][J][N]} is \var{PB[m][N]} where \var{J} is the number of
  servers at a multiserver.
\item[Q] \texonly{---} The \link{stations}{sec:server} in the network.  It is an
  array of pointers to stations and is indexed from 1 to \var{M}.
  \var{Q[0]} is not used.  
\item[U] \texonly{---} Station \underline{U}tilization.  Dimensioned by \bold{entry}
  and \bold{class}. 
\item[X] \texonly{---} the throughput for each \bold{class}.  It is
  valid for the last \link{population}{sec:population} vector solved
  (which is usually the maximum customer population.)
\item[Z] \texonly{---} The think time for each \bold{class}.
each time the MVA core step is executed.
\end{description}

\subsubsection{Variable Conventions}
The following variables are typically used throughout the MVA classes
either as arguments to, or locally within, a member function.

\label{sec:mva-conventions}
\begin{description}
\item[E] \texonly{---} The total number of entries for a station.
\item[e] \texonly{---} An entry index: \math[1 <= e <= E]{1 \leq e \leq E}.
\item[J] \texonly{---} The total number of servers at a multiserver
  station.
\item[j]  \texonly{---} A server index for multiservers: : \math[1 <=
  j <= J]{1 \leq j \leq J}.
\item[k] \texonly{---} A class (chain) index: \math[1 <= k <= K]{1
    \leq K \leq K}.
\item[m] \texonly{---} A station index: \math[1 <= m <= M]{1 \leq m
    \leq M}.
\item[N] \texonly{---} A Population \link{vector}{sec:vector}.
  Objects of type \link{\file{Population}}{sec:population} are indexed
  by this type.
\item[n] \texonly{---} A \emph{precomputed} index into the array that
  stores values for a given population \var{N}.
\end{description}


\subsubsection{Constructors for class MVA}

All solvers are created with the following type of call.  Subclasses
are expected to accept the same argument list.  The argument \var{m}
specifies the number of stations in the network.  \var{k} specifies
the number of classes.  \var{q} an array of pointers to objects of
type Server.  These objects are the stations in the queueing network.
Last, the vector \var{N} defines the maximum number of customers in
each class \var{k}.  Index 0 (zero) is NOT used in either \var{q} or
\var{N}.  Range checking does not take place for array \var{q}.

\begin{verbatim}
MVA( unsigned m, unsigned k, Server **q, Vector& N );
\end{verbatim}

\subsubsection{Instance Variable Access}

\begin{description}
\item[thinkTime] \texonly{---} Think Time.\\
  \code{double& thinkTime( const unsigned k );}

  The instance variable \var{Z}, (think time) can be set or retrieved
  with this function.
\end{description}

\subsubsection{Queries}

\begin{description}
\item[throughput] \texonly{---} Throughput.\\
  \code{double throughput( const unsigned e, const unsigned k ) const;}\\
  \code{double throughput( const unsigned e ) const;}

  Return the station's throughput for a given \bold{entry} and
  [optionally] \bold{class}.

\item[utilization] \texonly{---} Return the utilization of the station.\\
  \code{double utilization( const Server& station, const PopVector& N,
    const unsigned j ) const;} \\
  \code{double utilization( const Server& station, const unsigned k,
    const PopVector& N, const unsigned j ) const;} \\
  \code{double utilization( const unsigned m, const unsigned k, const
    PopVector& N ) const;} 

  The first of the three member functions returns the total
  utilization of \var{station} with one customer removed from class
  \var{j}.  The second member function returns the utilization for
  class \var{k} with one customer removed from class \var{j}.  The
  final version returns the utilization for class \var{k} at the full
  customer population \var{N}.

\item[sumOf_SL_m] \texonly{---} Queue length sum.\\
  \code{double sumOf_SL_m( const Server& s, const unsigned e, const
    PopVector &N, const unsigned j) const;}

  The function \code{sumOf_SL_m()} returns the sum of the service time
  multiplied by the queue length at station \var{s}, entry \var{e}, at
  the \link{population}{sec:population} \var{N} when one customer is
  removed from class \var{j}; i.e., \math[]{\sum_k s_{mk} L_m({\bf
      N}-e_{j})}.  This function is called from the
  \link{\file{Server}}{sec:server} class.  Its implementation depends on the
  MVA solver class.

\item[sumOf_p] \texonly{---} \math[Sum of Pi]{\sum (J_i-1-j)P_i(j,{\bf N} - e_j)}.\\
  \code{double sumOf_p( const Server& station, const PopVector &N, const
    unsigned j ) const}

  Return the \emph{something} with one customer removed from class
  \var{j} for population \var{N}. (See \link{Eqn. (2.19) and
    (3.8)}[~\Cite]{queue:reiser-80}, \link{Eqn.
    (2.1)}[~\Cite]{queue:conway-88}.)

\item[PB] \texonly{---} Pr(All Servers Busy)\\
  \code{double PB( const Server& station, const PopVector &N, const unsigned j ) const;}

  Return the probability that all servers are busy for a population of
  \var{N} and one customer removed from class \var{j}.

  Note: \math[PB(N) = P(J,N)]{{\it PB}_{m}({\bf N}) = P_{m}(J,{\bf N})}

\item[queueLength] \texonly{---} Find queue length.\\ \code{double
    queueLength( const Server&, const unsigned k, const PopVector& N,
    const unsigned j ) const;}

  Find the queue length at the server for class \var{k} with one customer
  from class \var{j} removed from the population \var{N}.  This function
  \emph{does not include} customers in service at the station. 

\item[iterations] \texonly{---} Number of core steps.\\
  \code{unsigned long iterations() const;}

  Returns the number of times the core MVA
  \link{\code{step}}{sec:mva-step} is called.

\item[offset] \texonly{---} Array offset calculation.\\
  \code{virtual unsigned offset( const PopVector& N ) const;} \\
  \code{virtual unsigned offset_e_j( const PopVector& N, const unsigned
    j ) const;} \\

  Calculate and return the \link{internal array
    offset}{sec:population-offset} for the
  \link{population}{sec:population} vector \var{N}.  These functions are
  defined by subclasses so that the proper offset function in the
  \link{\file{Population}}{sec:population} class can be called.  (Exact
  MVA uses \link{general populations}{sec:general-population} while the
  other solvers use \link{special populations}{sec:special-population}).

\end{description}

\subsubsection{Computation}
\begin{description}
\label{sec:mva-solve}
\item[solve] \texonly{---} Solve the model\\
\code{void solve();}

\code{Solve()} is used to solve the queueing network.  It invokes the
fuction \link{\code{step()}}{sec:mva-step} to perform the actual MVA
calculation.  Implemented by subclasses.


\label{sec:mva-step}
\item[step] \texonly{---} Core MVA step.\\
\code{void step( const PopVector& N );}

The \code{step()} member function is used to compute the queue lenght
and untilization at each station in the queueing network for the
population \var{N}.  It assumes that solution for the network exists
with one customer removed from each class \var{k}.  It first finds the
waiting time at each station (See Eqn.
\link{(1)}[~\Cite]{queue:chandy-82}). Next, the throughputs for each
class, \math[\var{X[k]}]{X_k}, are found.  This result is then used to
find the new queue length \math[\var{L}]{L_{mk}({\bf N})} (See Eqn.
\link{(6)}[~\Cite]{queue:chandy-82})

\begin{iftex}
\tex
\begin{equation}
  L_{mk}({\bf N}) = N_k \frac{ v_{mk} W_{mk}({\bf N}) }{
    \sum_{m^\prime} v_{m^\prime k} W_{m^\prime k}({\bf N})}
\end{equation}
\end{iftex}

The waiting time \math[\var{W}]{W_{mk}} is calculated by the class
\link{\code{Server}}[~(\Sec\Ref)]{sec:server}.  It supplies two
member functions, \link{\code{initStep()}}{sec:server-initstep} and
\code{wait()}.  \code{InitStep()} is called first and is used to
initialize any data prior to finding the new waiting time.
\code{Wait()} is used to find the waiting time at the current
population value.  Both functions are overridden by subclasses of
\file{Server} to provide the necessary functionality.
\begin{iftex}
  For example, Equation~\ref{eqn:delay} is used to find the waiting
  time for delay servers, Equation~\ref{eqn:fcfs} is used for First
  Come, First Served servers, and Equation~\ref{eqn:msfcfs} is used
  for multi-servers.
\tex
\begin{eqnarray}
W_{mk}({\bf N}) & = & s_{mk} \label{eqn:delay} \\
W_{mk}({\bf N}) & = & s_{mk} [ 1 + L_{mk}({\bf N} - e_k) ]
\label{eqn:fcfs} \\
W_{mk}({\bf N}) & = & \frac{ s_{mk} }{J_{m}} \left[ 1 + L_{mk}({\bf N} -
e_k) + \sum_{j=0}^{J_{m} - 2} (J_{m} - 1 - j) P(j,{\bf N} - e_k) \right]
\label{eqn:msfcfs} 
\end{eqnarray}
\end{iftex}

Note that \code{step()} may be called multiple times for a given
population (in particular, by the approximate solvers), while
\code{initStep()} is called only once.

\item[utilization] \texonly{---} Find utilization\\
\code{double utilization( const unsigned m, const unsigned e, const
  unsigned k, const double L ) const;}

Compute the utilization \math[\var{U}]{U_{mk}} for the various
stations and customer populations.  (See \link{Eqn.
  (7)}[~\Cite]{queue:chandy-82})

\begin{iftex}
\tex
\begin{equation}
  U_{mk}({\bf N}) = \frac{s_{mk} L_{mk}({\bf N})}{W_{m^\prime k}({\bf N})}
\end{equation}
\end{iftex}

\item[marginalProbabilities] \texonly{---} Compute P.\\
\code{void marginalProbabilities( const unsigned m, const PopVector& N );}

Compute the marginal probability that there are \var{j} customers at
station \var{m} for population \var{N}. The probabilities are stored
in \link{\var{P[m][j][N]}}{sec:mva-ivars}.  \bold{Note:} The
probability that all servers are busy is denoted as \var{PB} in the
literature, but is stored in \var{P} in the expected place (i.e.
\math[\var{j} = \var{M}]{j = J}).  (See Equations \link{(2.10) and
  (2.13) and (3.9)}[~\Cite]{queue:reiser-80} and \link{(2.4), (2.5)
  and (2.6)}[~\Cite]{queue:conway-88}).

\begin{iftex}
\tex
\begin{eqnarray}
P_{m}(j,{\bf N}) & = & \sum_{k=1}^{K} s_{mk}\lambda_{mk}({\bf N}))
P_{m}(j-1,{\bf N} - e_k) / j, \forall j, 0 < j < J \\
P_{m}(J,{\bf N}) & = & \sum_{k=1}^{K} s_{mk}\lambda_{mk}({\bf N}))
\left[ P_{m}(J,{\bf N} - e_k) + P_{m}(J-1,{\bf N} - e_k) \right] / J \\
P_{m}(0,{\bf N}) & = & 1 - \sum_{j=1}^{J} P_{m}(j,\bf{N})
\end{eqnarray}

Note that $P_{m}(J,{\bf N}) = PB_m({\bf N})$.
\end{iftex}

\end{description}

\subsubsection{Printing}
\begin{description}
\item[print] \texonly{---} Print all results.\\ \code{ostream& print(
    ostream& os );}

  Print the throughput and utilization for all
  \link{stations}{sec:server} using the
  \link{\code{printResults}}{sec:server-print} function.

\item[printL] \texonly{---} Print Queue Lengths.\\ \code{ostream&
    printL( ostream& os, const PopVector& N );}

  Print the value of queue length \link{\var{L}}{sec:server-ivars} at
  the \link{population}{sec:population} \var{N}.  This function is
  used to debug the solvers.

\item[printP] \texonly{---} Print marginal probabilities.\\ 
  \code{ostream& printZ( ostream& os );}

  Print marginal probabilities for all classes.

\item[printW] \texonly{---} Print waiting times.\\ \code{ostream&
    printL( ostream& os );}

  Print the value of waiting time \link{\var{W}}{sec:server-ivars}.
  This function is used to debug the solvers.

\item[printX] \texonly{---} Print throughputs.\\ 
  \code{ostream& printZ( ostream& os );}

  Print throughputs for all classes.

\item[printZ] \texonly{---} Print think times.\\ 
  \code{ostream& printZ( ostream& os );}

  Print think times for all classes.

\end{description}


\subsection{Class ExactMVA}
\label{sec:mva-exact}

The Exact MVA algorithm is taken from \link{Chandy and
  Neuse}[~\Cite]{queue:chandy-82} (Also refer to \link{Reiser and
  Lavenberg}[~\Cite]{queue:reiser-80} and \link{Lazowska et.
  al.}[~\Cite]{perf:lazowska:84}).  It sequences through all customer
populations, starting at \math[0,0,...0]{0_1,0_2,...,0_K}, calling the
\link{\code{step()}}{sec:mva-step} function for each.

\subsubsection{Queries}
\begin{description}
\item[offset] \texonly{---} Return array offset for population\\ 
  \code{unsigned offset( const PopVector& N ) const}\\ \code{unsigned
    offset_e_j( const PopVector& N, const unsigned j ) const}

  Return \link{population}{sec:population} array offset.

\end{description}
\subsubsection{Computation}
\begin{description}
\item[solve] \texonly{---} Solve model\\ \code{void solve()};

  Recursively solve for population vector \var{N} starting at a
  population of \math[0,0,...,0]{[0_1,0_2,...0_K]} to \var{nCust}.
  Solutions for each \link{population}{sec:population} vector in the
  population space are generated by calling \code{nextPop()}.  The
  dimensionality of \var{N} is limited by stack size.

Note: while is is possible to save space by not retaining
\link{population}{sec:population} values for the ``outermost'' class
(see \link{Lazowska}[~\Cite]{perf:lazowska:84}, Chapter 19), no
attempt is done so here.

\end{description}

\subsection{Class Schweitzer}
\label{sec:mva-schweitzer}

\subsubsection{Instance Variables}
\label{sec:mva-schweitzer-ivars}
\begin{verbatim}
private:
    double ***last_L;               /* For local comparison.        */
\end{verbatim}

\begin{description}
\item[last_L] \texonly{---} Previous value of the queue length \var{L}
  and is used to test for the termination condition.
\end{description}

\subsubsection{Initialization}
\begin{description}
\label{sec:mva-schweitzer-init-L}
\item[init_L] \texonly{---} Initialize Queue Lengths\\
\code{virtual void init_L( const PopVector & N);}

Find the queue length based on fraction \var{F} of class \var{k} jobs
at station for population \var{N} (See \link{Eqn.
  (11)}[~\Cite]{queue:chandy-82}).

\begin{iftex}
\tex
\begin{equation}
L_{mk}({\bf N} - e_j) = ({\bf N} - e_j)F_{mk}({\bf N}) + D_{mk}(\bf N)
\end{equation}
\end{iftex}

\item[init_U] \texonly{---} Initialize Utilizations\\
\code{virtual void init_U( const PopVector & );}

Initialize the utilization array \math[\var{U}]{U \forall m,e,k} for
population \math[\var{N} with one customer removed from each class in
turn]{N - e_j, 1 \leq j \leq K}.  \code{init_U()} is separated from
\link{\code{init_L()}}{sec:mva-schweitzer-init-L} to permit subclasses to override.

\end{description}
        
\subsubsection{Queries}
\begin{description}
\item[D_mekj] \texonly{---} Return D\\
\code{virtual double D_mekj(unsigned m,unsigned e,unsigned k,unsigned j);}

Return the change in the fraction of class \var{k} jobs at queue
\var{m} after one class \var{j} job has been removed.  For the
Schweitzer solver, the best estimate of \var{D} is zero.

\end{description}
\subsubsection{Computation}

\begin{description}
\item[solve] \texonly{---} Solve model using Schweitzer method.\\
\code{virtual void solve();}

\code{solve()} merely is a public interface to \code{core()}.

\label{sec:mva-schweitzer-core}
\item[core] \texonly{---} \\
\code{void core( const PopVector & );}

\code{Core()} is the basic iterative solver for Bard Schweitzer and
Linearizer algorithms.  For each iteraction, it first calls
\link{\code{init_L()}}{sec:mva-schweitzer-init-L} and \code{init_U()}
followed by \link{\code{step()}}{sec:mva-step}.

\end{description}

\subsection{Class Linearizer}
\label{sec:mva-linearizer}

The \link{linearizer
  algorithm}[~\Cite]{queue:chandy-82,queue:krzesinski-84,queue:conway-88}
improves upon the \link{Schweitzer}{sec:mva-schweitzer} method by
providing better estimates for the variable \math[\var{D}]{D_{mejk}}.
Linearizer calls the Schweitzer solver \math[K+1]{K+1} times per outer
loop: once each with one customer from class \math[\var{c}]{c}
removed, and once at the maximum customer population.  The outer loop
is executed three times.

\subsubsection{Instance Variables}
\label{sec:mva-linearizer-ivars}
\begin{verbatim}
protected:
    Population ***F;            /* Fraction of jobs at queue.   */
    Population ***saved_L;      /* Saved queue length.          */
    Population ***saved_U;      /* Saved utilization.           */
    double ****D;               /* Delta Fraction of jobs.      */
\end{verbatim}

\begin{description}
\item[D] \texonly{---} The change in the fraction of queue length for
  station \var{m}, entry \var{e}, class \var{k} when a class \var{j}
  customer is removed.
\item[F] \texonly{---} The fraction of class \var{k} jobs at station
  \var{m}, entry \var{e}. 
\item[saved_L] \texonly{---} Saved queue lengths.
\item[saved_U] \texonly{---} Saved station utilizations.
\end{description}

\subsubsection{Initialization}
\begin{description}
\label{sec:mva-linearizer-init-D}
\item[init_D] \texonly{---} Intialize \var{D}\\
\code{virtual void init_D( const PopVector & N );}

Initialize \var{D}, the difference in queue lengths when a customer
from class \var{j} is removed (See Eqn.
\link{(10)}[~\Cite]{queue:chandy-82}).  This member function is
overridden for the \link{fast linearizer}{sec:mva-fast} solver.
\begin{iftex}
\tex
\begin{equation}
D_{mkj}({\bf N}) = F_{mk}({\bf N} - e_j) - F_{mk}({\bf N})
\end{equation}
\end{iftex}

\end{description}

\subsubsection{Queries}

\begin{description}

\item[D_mekj] \texonly{---} Return D\\
code{double D_mekj( unsigned m, unsigned e, unsigned k, unsigned j );}

Return the change in the fraction of class \var{k} jobs at queue
\var{m} after one class \var{j} job has been removed.  This function
overrides the member function supplied by the \file{Schweitzer} class because
\var{D} is now computed.

\end{description}
        
\subsubsection{Computation}

\begin{description}
\item[solve] \texonly{---} Solve model using Linearizer method.\\
\code{virtual void solve();}

Solve the model using the linearizer solution strategy.  This member
function calls the Schweitzer solver,
\link{\code{core()}}{sec:mva-schweitzer-core} with one customer
removed from each class in turn and at the full population value.  The
instance variable \link{\var{c}}{sec:mva-ivars}, defined in the
abstract superclass, is set according to the class in which the
customer is removed.

The member function \link{\code{init_D()}}{sec:mva-linearizer-init-D}
is called to initialize the \var{D} array before each call to the
Schweitzer \code{core()} solver.  

\end{description}

\subsubsection{Input and Output}
\begin{description}
\item[print] \texonly{---} Printing fuctions\\
\code{ostream& printF( ostream&, const PopVector& N );}\\
\code{ostream& printD( ostream&, const PopVector& N );}

Print out the value of the instance variables \var{F} and \var{D} for
the population \var{N} respecitively.  Used for debugging only.

\end{description}

\subsection{Class Linearizer2}
\label{sec:mva-fast}

Algorithm is from \link{de Souza e Silva and Muntz}[~\Cite]{queue:deSouzaeSilva-90}.

\subsubsection{Instance Variables}
\label{sec:mva-fast-ivars}
\begin{verbatim}
private:
    SpecialPop **Lm;            /* Queue length sum (Fast lin.) */
    SpecialPop **Um;            /* Queue length sum (Fast lin.) */
    double ***D_k;              /* Sum over k.                  */
\end{verbatim}

\begin{description}
\item[D_k] \texonly{---} \math[Sum over class \var{k} of
  \var{D}]{D^\prime_{mj} = \sum_{k=1}^{K} N_k D_{mkj}({\bf N})}.
\item[Lm] \texonly{---} Queue \underline{L}ength summed over all stations.  Used by
  the \link{fast linearizer}{sec:mva-fast}.  Dimensioned by \bold{entry}.
\item[Um] \texonly{---} Station \underline{U}tilization.  Used by the \link{fast
    linearizer}{sec:mva-fast}.  Dimensioned by \bold{entry}. 
\end{description}
\subsubsection{Initialization}

\begin{description}
\label{sec:mva-fast-init-D}
\item[init_D] \texonly{---} Initialize D array.\\
\code{void init_D( const PopVector & );}

Initialize \var{D}, the difference in queue lengths when a customer
from class \var{j} is removed.  This version of the \code{init_D()}
member function also initializes \math[\var{D_k}]{D^\prime_{mj}}.


\label{sec:mva-linearizer-init-L}
\item[init_L] \texonly{---} Initialize Queue lengths.\\
\code{void init_L( const PopVector & );}

Pre-compute \math[\var{S} times \var{Lm}]{S \cdot L_{mk}} and save.
\math[\var{S} times \var{L}]{S \cdot L_{mk}} must be pre-computed
because it relies on the old values of \math[\var{Lm}]{L_{mk}} for all
populations of \math[\var{N}]{\bf N}.  The expressions from \link{de
  Souza e Silva and Muntz}[~\Cite]{queue:deSouzaeSilva-90} (Equations
(8) and (9)) do not multiply through by service times --- this version
does to simplify the expressions for the \link{High Variability First
  Come, First Served}{sec:server-hvfcfs} Server.

\begin{iftex}
\tex
\begin{eqnarray}
L_{m}({\bf N} - e_j ) & = & L_{m}({\bf N}) - \frac{L_{m}({\bf N})}{N_j} +
D^\prime_{mj}({\bf N}) - D_{mjj}({\bf N}) \\
L_{m}({\bf M} - e_j ) & = & L_{m}({\bf M}) - \frac{L_{m}({\bf M})}{M_j} +
D^\prime_{mj}({\bf M}) - D_{mjj}({\bf M}) + - D_{mcj}({\bf M}) \\
{\bf M} & = & {\bf N} - e_c \nonumber
\end{eqnarray}
\end{iftex}

\item[init_U] \texonly{---} Initialize Utilizations.\\
\code{void init_U( const PopVector & );}

Initialize the utilization array \math[\var{U}]{U \forall m,e,k} for
population \math[\var{N}]{N - e_j, 1 \leq j \leq K}.

\end{description}
\subsubsection{Queries}

\begin{description}

\item[sumOf_SL_m] \texonly{---} Queue length sum.\\
\code{double sumOf_SL_m( const Server& s, const unsigned e, const
  PopVector &N, const unsigned j ) const;}

Return the pre-computed value of \var{Lm} for entry \var{e}, class
\var{j}.  This value was calculated in
\link{\code{init_L()}}{sec:mva-linearizer-init-L} earlier.

\end{description}

\C Local Variables: 
\C mode: latex
\C TeX-master: "lqns"
\C End: 
