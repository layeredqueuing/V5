%%  -*- mode: latex; mode: outline-minor; fill-column: 108 -*-
%% Title:  defects
%%
%% $HeadURL: http://rads-svn.sce.carleton.ca:8080/svn/lqn/trunk/doc/userman/defects.tex $
%% Original Author:     Greg Franks <greg@sce.carleton.ca>
%% Created:             Tue Jul 18 2006
%%
%% ----------------------------------------------------------------------
%% $Id: defects.tex 7586 2007-08-02 11:13:34Z greg $
%% ----------------------------------------------------------------------

\chapter{Known Defects}
\label{sec:defects}

\section{MOL Multiserver Approximation Failure}
\label{sec:MOLMultiserver}

The MOL multiserver approximation sometimes fails when the service
time of the clients to the multiserver are significantly smaller than
the service time of the server itself.  The utilization of the
multiserver will be too high.  Sometimes, the problem can be solved by
changing the mol-underrelaxation.  Otherwise, switch to the
more-expensive Conway multiserver approximation.

\section{Chain construction for models with multi- and infinite-servers}
\label{sec:ChainConstruction}

\section{No algorithm for phased multiservers OPEN class.}
\label{sec:PhaseMultiOpen}

\section{Overtaking probabilities are calculated using CV=1}
\label{sec:Overtaking}

\section{Need to implement queue lengths for open classes.}
\label{sec:QueueLengths}



%%% Local Variables: 
%%% mode: latex
%%% mode: outline-minor 
%%% fill-column: 108
%%% TeX-master: "userman"
%%% End: 
