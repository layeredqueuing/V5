\C 
\C $Id: introduction.tex 5439 2005-01-02 02:30:31Z greg $
\C 
\C $Log$
\C Revision 31.0  2005/01/02 02:30:29  greg
\C Bring all files up to revision 31.
\C
\C Revision 30.0  2003/11/12 18:56:52  greg
\C Bring all files up to revision 30.0
\C
\C Revision 29.0  2003/06/13 00:48:38  greg
\C bounce version number.
\C
\C Revision 1.3  1995/01/11 21:05:12  greg
\C Revised for R12 of MVA software.
\C
\C Revision 1.2  1994/12/09  02:02:36  greg
\C update.
\C
\C Revision 1.1  1994/06/23  19:11:01  greg
\C Initial revision
\C
\C ----------------------------------------------------------------------
\chapter{Introduction}

The {\bf L}ayered {\bf Q}ueueing {\bf N}etwork {\bf Solver} (LQNS) is
a software package for solving \link{\emph{``Stochastic Rendevous
    Networks''}}[~\Cite]{srvn:woodside-89b} or \link{\emph{``Layered
    Queueing Models''}}[~\Cite]{perf:rolia-92}.  These models are
characterized by multiple layers of \link{\dfn{tasks}}{sec:task} using
the send-receive-reply or rendezvous mechanism for inter-task
communication.  Services provided by each task in the model are
further specified though \link{\dfn{entries}}{sec:entries}.  Service
requests from one task to another are made through
\link{\dfn{Calls}}{sec:call}, which in fact link one entry to
another.

The sections that follow show the overall data flow through the
solver, and how the major components used in generating a solution are
connected together.

\section{Data Flow}
\label{sec:data-flow}

Data flow...

\begin{figure}[htbp]
  \label{fig:dataflow}
  \begin{center}
    \T \tex \leavevmode \input{dataflow.epic}
    \caption{Data Flow}
    \H \htmlimage{dataflow.gif}
  \end{center}
\end{figure}

\section{Class Hierarchy}
\label{sec:class-hierarchy}

The \link{overall class hierarchy}{fig:overall} for the LQNS solver is
shown below \texonly{in Figure~\ref{fig:overall}} using the notation
of \link{Rumbaugh et.~al.}[~\Cite]{sw:rumbaugh-91}.  The first row of
classes, \file{Load}, \file{Store} and \file{Layerize} correspond to
the processes in the second dataflow diagram in
\texonly{Figure~\ref{fig:dataflow}}\htmlonly{\link{the last
    section}{fig:dataflow}}. The classes \file{Call}, \file{Entry} and
\file{Entity} are used to created objects that represent the
\file{Model Database}.  Finally, the classes \file{MVA} and
\file{Server} are used to solve the MVA models created the the class
\file{Layerize}.

\begin{figure}[htbp]
  \label{fig:overall}
  \begin{center}
    \T \tex \leavevmode \input{overall.epic}
    \caption{Class Hierarchy}
    \H \htmlimage{overall.gif}
  \end{center}
\end{figure}

\C Local Variables: 
\C mode: latex
\C TeX-master: "lqns"
\C End: 
